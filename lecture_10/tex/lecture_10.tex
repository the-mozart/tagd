\documentclass{beamer}

\usepackage{beamerthemesplit}
\usetheme{Singapore} %Copenhagen}
%\usecolortheme{whale}

%\usepackage[T2A]{fontenc}
%\usepackage[utf8]{inputenc}
%\usepackage[russian]{babel}

\usepackage[main=russian,english]{babel}   %% загружает пакет многоязыковой вёрстки
\usepackage{fontspec}      %% подготавливает загрузку шрифтов Open Type, True Type и др.
\defaultfontfeatures{Ligatures={TeX},Renderer=Basic}  %% свойства шрифтов по умолчанию
\setmainfont{Times New Roman} %% задаёт основной шрифт документа
%\usefonttheme{professionalfonts}% SOLUTION
\usefonttheme{serif}

\usepackage{hyperref}
\usepackage{textcomp}
\usepackage{amssymb,amsmath}
%\usepackage{animate}
%\usepackage{longtable}
\usepackage{xcolor}

%\usepackage{pgffor}
\usepackage{enumitem}


\newcounter{N}

%% Форматирование окружения itemize
%\usepackage{ragged2e}
%\let\olditem\item
%\renewcommand\item{\olditem\justifying}

\usepackage{ mathrsfs }
\newcommand{\Rho}{\mathscr{P}}

\newcommand{\argxi}{(\xi^1,\xi^2,\xi^3)}
\newcommand{\argx}{(x^1,x^2,x^3)}

\newcommand{\argxiv}{(\vec{\xi})}
\newcommand{\argxv}{(\vec{x})}


\newcommand{\argxbarn}{(\bar{x}^1,\bar{x}^2,\ldots, \bar{x}^n)}
\newcommand{\argxn}{(x^1, x^2,\ldots, x^n)}

\newcommand{\argtxi}{(t, \xi^1,\xi^2,\xi^3)}
\newcommand{\argtoxi}{(t_0, \xi^1,\xi^2,\xi^3)}

\newcommand{\argtxiv}{(t, \vec{\xi})}
\newcommand{\argtoxiv}{(t_0, \vec{\xi})}


\newcommand{\argtx}{(t, x^1,x^2,x^3)}
\newcommand{\argtox}{(t_0, x^1,x^2,x^3)}

\newcommand{\argtxv}{(t, \vec{x})}
\newcommand{\argtoxv}{(t_0, \vec{x})}


\newcommand{\pd}[2]{\frac{\partial #1}{\partial #2}}
\newcommand{\pdk}[2]{\frac{\partial^2 #1}{\partial #2^2}}

\newcommand{\grad}{\operatorname{grad}}
\newcommand{\rot}{\operatorname{rot}}
\newcommand{\divo}{\operatorname{div}}

\title[]{Общая теория движения жидких и газообразных сред}

\author[]{ {\em Верещагин Антон Сергеевич}
\\
канд. физ.-мат. наук, старший преподаватель\\
\bigskip
Кафедра аэрофизики и газовой динамики ФФ НГУ}

\usebackgroundtemplate{\includegraphics[width=\paperwidth]{../img/background.png}}

\begin{document}
	
\frame{\titlepage}


\frame{
	\frametitle{Аннотация}
	\parbox{\textwidth}{

	}
}


\frame{
	\frametitle{ Модель баротропного течения идеального газа }
	
	\begin{exampleblock}{ Основные уравнения }
		\parbox{\textwidth}{
			\[
			\pd{\rho}{t} + \operatorname{div}(\rho \vec{v}) = 0,
			\]
			\[
			\pd{\vec{v}}{t} + (\vec{v}\cdot\nabla) \vec{v} = -\frac{1}{\rho} \nabla p + \vec{f},
			\]
			где $\rho$, $p$, $\vec{v}$ -- плотность, давление и скорость среды, заданные в Эйлеровой системе координат ($x$, $y$, $z$); $\vec{f}$ -- вектор объёмных сил.
			
		}
	\end{exampleblock}
	\begin{exampleblock}{Определение}
		\parbox{\textwidth}{
			Течение называется \alert{баротропным}, если между плотностью и давлением имеет место соотношение 
			\[
				p = p(\rho).
			\]
					
		}
	\end{exampleblock}	
}

\frame{
	\frametitle{Примеры баротропных течений}
	\begin{itemize}[partopsep=1pt,label=\textbullet]
	\item  Изотермические течения
			\[
				p = \rho R T,
			\]
			где $R$ -- газовая постоянная; $T$ -- заданная температура.
			
	\item Изоэнтропические течения политропного газа
		\[
		p = A(S) \rho^\gamma,
		\]
		где $S$ -- энтропия; $\gamma=C_p/C_V$ -- показатель политропы.
		
	\item Идеальная жидкость
			\[
				\rho = const.
			\]
			
	
	\end{itemize}

}

\frame{
	\frametitle{Преобразование конвективной части закона движения }
	
	\begin{exampleblock}{Предпосылки}
		\parbox{\textwidth}{
			Из формулы 
			\[
			\operatorname{grad} (\vec{a}\cdot\vec{b}) = (\vec{b}\cdot\nabla)\vec{a}+
			(\vec{a}\cdot\nabla)\vec{b}+
			\vec{b} \times \operatorname{rot} \vec{a} +
			\vec{a} \times \operatorname{rot} \vec{b}
			\]
			следует, что
			\[
			\operatorname{grad} 
			\left( \frac{\vec{v}^2}{2} \right) = (\vec{v} \cdot \nabla) \vec{v} +
			\vec{v}\times \operatorname{rot} \vec{v}.
			\]
			
		}
	\end{exampleblock}
	
	\begin{exampleblock}{Модификация уравнений}
		\parbox{\textwidth}{
			\[
			\pd{\vec{v}}{t} + \operatorname{grad} \left( \frac{\vec{v}^2}{2} \right) + \operatorname{rot} \vec{v} \times \vec{v} = \vec{f} - \frac{1}{\rho} \operatorname{grad} p.
			\]
		}
	\end{exampleblock}
	
}

\frame{
	\frametitle{ Функция давления $\Rho(p)$ }
	\begin{exampleblock}{Определение}
		\parbox{\textwidth}{
			Для баротропного течения, если существует $\rho=\rho(p)$, то определим
			\[
				\Rho(p) = \int\limits_{p_0}^{p} \frac{dp}{\rho}.
			\]
		}
	\end{exampleblock}

	\begin{exampleblock}{Свойство}
		\parbox{\textwidth}{
			Используя соотношения аналогичные
			\[
				\pd{}{x}\Rho(p) = \pd{\Rho}{p} \pd{p}{x} = \frac{1}{\rho}  \pd{p}{x},
			\]
			получим
			\[
				\grad \Rho(p) =  \frac{1}{\rho} \grad p.
			\]
			
		}
	\end{exampleblock}
}


\frame{
	\frametitle{ Форма Громеки-Ламба уравнения движения }
	
	\begin{exampleblock}{Основные уравнения}
		\parbox{\textwidth}{
			Используя преобразование конвективной составляющей уравнения движения и  введённую функцию давления получим уравнения движения в форме Громеки-Ламба
			\[
				\pd{\vec{v}}{t} + \grad E +
				\vec{\Omega} \times \vec{v} = 0,
			\]
			где
			\[
				E = \frac{\vec{v}^2}{2} + \Rho + \Pi,\quad
				\vec{\Omega} = \rot \vec{v}.
			\]
			
			Здесь $\Pi$ -- потенциал массовых сил $\vec{f} = -\grad \Pi$.
		}
	\end{exampleblock}

	\begin{exampleblock}{Определение}
		\parbox{\textwidth}{
			$E$ -- представляет собой полную приведённую механическую энергию системы,
			$\vec{\Omega}$ -- вектор вихря.
		}
	\end{exampleblock}
	
}

\frame{
	\frametitle{ Уравнение динамической возможности движения }
	
		\parbox{\textwidth}{
			Из уравнения в форме Громеки-Ламба следует, что
			\[
			\pd{\vec{v}}{t} + 			\vec{\Omega} \times \vec{v} = -\grad E,
			\]
			поэтому
			\[
			\rot \left(\pd{\vec{v}}{t} + \vec{\Omega} \times \vec{v} \right) = 0.
			\]
		}

		\parbox{\textwidth}{
			Зная разложение для ротора векторного произведения, равного
			\[
				\rot(\vec{\Omega} \times\vec{v}) = (\vec{v}\cdot\nabla)\vec{\Omega} -
				(\vec{\Omega}\cdot\nabla) \vec{v}+
				\vec{\Omega} \divo \vec{v} -
				\vec{v} \divo \vec{\Omega},
			\]
			получим, используя полную производную, \alert{уравнение динамической возможности движения}
			\[
			\frac{d\vec{\Omega}}{dt} = (\vec{\Omega}\cdot\nabla) \vec{v}-
			\vec{\Omega} \divo \vec{v}.
			\]
			
		}
	
}

\frame{
	\frametitle{ Постоянство $E$ вдоль линий тока и вихревых линий }


	\parbox{\textwidth}{
		Пусть $\displaystyle\pd{\vec{v}}{t}=0$, тогда, умножив уравнение движения в форме Громеки-Ламба скалярно на вектор скорости $\vec{v}$, получим
		\[
		\vec{v} \cdot \grad E + \vec{v} \cdot (\vec{\Omega}\times \vec{v}) = 0.
		\]
		
		Второе слагаемое равно нулю, в силу определения векторного произведения, а первое является производной от $E$ вдоль линии тока $\vec{r} = \vec{r}(s)$, таких что $\displaystyle\pd{\vec{r}}{s} = \vec{v}$:
		\[
		\grad E \cdot  \vec{v} \cdot = \grad E \cdot \pd{\vec{r}}{s} = \pd{E}{s} = 0.
		\]
		
		Аналогично, умножая уравнение движения скалярно на вектор $\vec{\Omega}$, можно показать, что функция $E$ постоянна вдоль вихревых линий.
	}

	
}


\frame{
	\frametitle{Интеграл Бернулли для баротропного стационарного течения идеального газа}
	
	\begin{exampleblock}{Условия существования}
		\parbox{\textwidth}{
		\[
			p=p(\rho),\quad
			\pd{\vec{v}}{t} = 0.
		\]
			
		}
	\end{exampleblock}

	\begin{exampleblock}{Интеграл Бернулли}
		\parbox{\textwidth}{
			\[
			\frac{\vec{v}^2}{2} + \Rho(p) + \Pi = C(L),
			\]
			где $C(L)$ -- константа вдоль линии тока или вихревой линии; $\Rho(p)$ -- функция давления; $\Pi$~-- потенциал объёмных сил
			\[
			\Rho(p) = \int\limits_{p_0}^{p} \frac{dp}{\rho},\quad
			\vec{f} = -\nabla \Pi.
			\]
		}
	\end{exampleblock}
	
}

\frame{
	\frametitle{ Существование интеграла Бернулли во всей исследуемой области }
	
	\parbox{\textwidth}{
	Пусть в исследуемой области $\vec{\Omega} \times \vec{v} = \vec{0}$ и движение стационарно, тогда автоматически выполняется условие
	\[
	\frac{\vec{v}^2}{2} + \Rho(p) + \Pi = const
	\]
	во всей области.
	}

	\begin{exampleblock}{$\vec{\Omega} \times \vec{v} = \vec{0}$}
		\parbox{\textwidth}{
			\begin{itemize}[partopsep=1pt,label=$\rightarrow$]
				\item  $\vec{v} = 0$ -- покоящееся течение (гидростатика).
				\item $\vec{\Omega}=\rot \vec{v} = \vec{0}$ -- безвихревое или потенциальное течение.
				\item $\vec{\Omega} \parallel \vec{v}$ -- вихревые линии совпадают с линиями тока (винтовое течение).
				
			\end{itemize}
		}
	\end{exampleblock}
	
}

\frame{
	\frametitle{ Баротропное безвихревое течение }
	
	\begin{exampleblock}{Определение}
		\parbox{\textwidth}{
			Течение называется \alert{безвихревым} или \alert{потенциальным}, если 
			\[
				\vec{\Omega} = \rot \vec{v} = \vec{0}
			\]
			или 
			\[
				\vec{v} = \nabla \varphi,
			\]
			где $\varphi\argxv$ -- потенциал скорости. 
			
		}
	\end{exampleblock}

	\begin{exampleblock}{Уравнение движения в форме  Громеки-Ламба для потенциального течения}
		\parbox{\textwidth}{
			
		\[
			\grad \left( \pd{\varphi}{t} + \frac{(\nabla\varphi)^2}{2} + \Rho + \Pi\right) = 0,
		\]
		
		 Слагаемое $\vec{\Omega} \times \vec{v}=0$ в силу того, что $\vec{\Omega}=\vec{0}$.
		}
	\end{exampleblock}
	
}


\frame{
	\frametitle{ Интеграл Коши для баротропного потенциального течения идеального газа }
	
		\begin{exampleblock}{Условия существования}
		\parbox{\textwidth}{
			\[
			p=p(\rho),\quad
			\vec{v} = \nabla\varphi.
			\]
			
		}
	\end{exampleblock}
	
	\begin{exampleblock}{Интеграл Коши}
		\parbox{\textwidth}{
			\[
			\pd{\varphi}{t} + \frac{(\nabla\varphi)^2}{2} + \Rho + \Pi = F(t),
			\]
			где $F(t)$ -- постоянная функция во всей области, различающаяся в различные моменты времени; $\Rho(p)$ -- функция давления; $\Pi$ -- потенциал объёмных сил
			\[
			\Rho(p) = \int\limits_{p_0}^{p} \frac{dp}{\rho},\quad
			\vec{f} = -\nabla \Pi.
			\]
		}
	\end{exampleblock}
	
}

\frame{
	\frametitle{ Кинетическая энергия безвихревого стационарного течения  идеальной жидкости}
	\begin{exampleblock}{Определение}
		\parbox{\textwidth}{
			Рассмотрим ограниченный односвязный объем $\omega$, в котором реализуется потенциальное течение c потенциалом $\varphi$ идеального жидкости ($\rho=const$). Тогда кинетическая энергия этого объёма будет задаваться формулой
			\[
			T = \frac{1}{2}\int\limits_\omega\rho\vec{v}^2 d\omega=
			\frac{1}{2}\rho\int\limits_\omega\left(
			\left(\pd{\varphi}{x}\right)^2+
			\left(\pd{\varphi}{y}\right)^2+
			\left(\pd{\varphi}{z}\right)^2
			\right)d\omega.
			\]
		}
	\end{exampleblock}

}


\frame{
	\frametitle{Кинетическая энергия безвихревого стационарного течения идеальной жидкости  }
		\begin{exampleblock}{Формула Грина}
		\parbox{\textwidth}{
			\[
			\int\limits_\omega \left(
			\pd{\varphi}{x}\pd{\varphi'}{x}+
			\pd{\varphi}{y}\pd{\varphi'}{y}+
			\pd{\varphi}{z}\pd{\varphi'}{z}
			\right) d\omega=
			-\int\limits_S \varphi\pd{\varphi'}{n} dS  -
			\]
			\[
			-\int\limits_{\omega} \varphi \left(
			\pdk{\varphi'}{x} +\pdk{\varphi'}{y}+\pdk{\varphi'}{z}	
			\right) d\omega,
			\]
			где $S$ -- поверхность $\omega$; $\vec{n}$ -- вектор внешней единичной нормали.
		}
	\end{exampleblock}

	\begin{exampleblock}{Уравнение неразрывности}
		\parbox{\textwidth}{
			Подставляя в уравнение неразрывности $\vec{v} = \nabla \varphi$ и $\rho=const$, получим
			\[
			\Delta \varphi = \pdk{\varphi}{x}+\pdk{\varphi}{y}+\pdk{\varphi}{z} = 0.
			\]
		}
	\end{exampleblock}
}


\frame{
	\frametitle{ Кинетическая энергия безвихревого стационарного течения идеальной жидкости }


	\begin{exampleblock}{Кинетическая энергия}
		\parbox{\textwidth}{
			Используя формулу Грина и уравнение неразрывности получим
			\[
			T = -\frac{1}{2}\rho\int\limits_S\varphi\pd{\varphi}{n}dS.
			\]
			
			Таким образом, кинетическая энергия объёма зависит только от значений потенциала и его производной на границе. Если на границе реализуется условие непротекания  $\displaystyle\pd{\varphi}{n} = 0$ или потенциал постоянен (а он определяется с точностью до константы), то жидкость внутри односвязного объёма покоится.
			
		}
	\end{exampleblock}
}

\frame{
	\frametitle{ Теорема В. Томсона }
	
	\small
	
	\begin{exampleblock}{Теорема}
		\parbox{\textwidth}{
			Кинетическая энергия несжимаемой жидкости, движущейся в односвязном объёме с потенциалом скоростей, меньше кинетической энергиии во всяком другом движении, при котором на границах объёма жидкость обладает движением, одинаковым с безвихревым, внутри же обладает вихрями.
			
		}
	\end{exampleblock}

	\begin{exampleblock}{Математическая формулировка}
		\parbox{\textwidth}{
			Рассмотрим два стационарных течения идеальной жидкости с плотностью $\rho$, одно безвихревое с потенциалом $\varphi$, другое -- не потенциальное со скоростью $\vec{v}$. Рассмотрим односвязный объём $\omega$ с границей $S$, такой что на нем выполняется условие
			\[
			\vec{v}\cdot\vec{n}|_S = \left. \pd{\varphi}{n} \right|_S.
			\]
			Тогда $T'>T$, где $T$, $T'$ - кинетическая энергия объема $\omega$ для потенциального и вихревого течений соответственно.
		}
	\end{exampleblock}
}

\frame{
	\frametitle{ Теорема В. Томсона: доказательство }
	
}


\frame{
	\frametitle{ Литература }
	\begin{itemize}[partopsep=1pt,label=\textbullet]
		\item  %{\em Овсянников~Л.~В.} Лекции по основам газовой динамики.  Москва-Ижевск:Институт компьютерных исследований, 2003.	
		
	\end{itemize}
}



\end{document}