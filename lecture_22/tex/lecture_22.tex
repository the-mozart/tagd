\documentclass{beamer}

\usepackage{beamerthemesplit}
\usetheme{Singapore} %Copenhagen}
%\usecolortheme{whale}

\beamertemplatenavigationsymbolsempty % Hide navigation panel

%\usepackage[T2A]{fontenc}
%\usepackage[utf8]{inputenc}
%\usepackage[russian]{babel}

\usepackage[main=russian,english]{babel}   %% загружает пакет многоязыковой вёрстки
\usepackage{fontspec}      %% подготавливает загрузку шрифтов Open Type, True Type и др.
\defaultfontfeatures{Ligatures={TeX},Renderer=Basic}  %% свойства шрифтов по умолчанию
\setmainfont{Times New Roman} %% задаёт основной шрифт документа
%\usefonttheme{professionalfonts}% SOLUTION
\usefonttheme{serif}

\usepackage{hyperref}
\usepackage{textcomp}
\usepackage{amssymb,amsmath}
%\usepackage{animate}
%\usepackage{longtable}
\usepackage{xcolor}
\usepackage{icomma}

%\usepackage{pgffor}
\usepackage{enumitem}
\usepackage[export]{adjustbox}

\newcounter{N}

%% Форматирование окружения itemize
%\usepackage{ragged2e}
%\let\olditem\item
%\renewcommand\item{\olditem\justifying}

\usepackage{ mathrsfs }
\newcommand{\Rho}{\mathscr{P}}

\renewcommand{\Re}{\operatorname{Re}}
\newcommand{\Sh}{\operatorname{Sh}}
\newcommand{\Eu}{\operatorname{Eu}}
\newcommand{\Fr}{\operatorname{Fr}}

%\DeclareMathOperator{\tg}{tg}
\DeclareMathOperator{\сtg}{сtg}


\newcommand{\argxi}{(\xi^1,\xi^2,\xi^3)}
\newcommand{\argx}{(x^1,x^2,x^3)}

\newcommand{\argxiv}{(\vec{\xi})}
\newcommand{\argxv}{(\vec{x})}


\newcommand{\argxbarn}{(\bar{x}^1,\bar{x}^2,\ldots, \bar{x}^n)}
\newcommand{\argxn}{(x^1, x^2,\ldots, x^n)}

\newcommand{\argtxi}{(t, \xi^1,\xi^2,\xi^3)}
\newcommand{\argtoxi}{(t_0, \xi^1,\xi^2,\xi^3)}

\newcommand{\argtxiv}{(t, \vec{\xi})}
\newcommand{\argtoxiv}{(t_0, \vec{\xi})}

\newcommand{\argtxo}{(t, x)}
\newcommand{\arguv}{(\vec{u})}

\newcommand{\argtxv}{(t, \vec{x})}
\newcommand{\argtoxv}{(t_0, \vec{x})}


\newcommand{\pd}[2]{\frac{\partial #1}{\partial #2}}
\newcommand{\pdk}[2]{\frac{\partial^2 #1}{\partial #2^2}}

\newcommand{\od}[2]{\frac{d #1}{d #2}}
\newcommand{\odk}[3]{\frac{d^{#3} #1}{d #2^{#3}}}

\newcommand{\grad}{\operatorname{grad}}
\newcommand{\rot}{\operatorname{rot}}
\newcommand{\divo}{\operatorname{div}}

\newtheorem{dfn}{Определение}  
\newtheorem{theorems}{Теорема}  

\title[]{Решения со слабыми разрывами уравнений газовой динамики}

\author[]{ {\em Верещагин Антон Сергеевич}
\\
канд. физ.-мат. наук, старший преподаватель\\
\bigskip
Кафедра аэрофизики и газовой динамики ФФ НГУ}

\usebackgroundtemplate{\includegraphics[width=\paperwidth]{../img/background.png}}

\begin{document}
	
\frame{\titlepage}

\frame{
	\frametitle{Аннотация}
	\parbox{\textwidth}{

	}
}


\frame{
	\frametitle{Характеристики системы квазилинейных уравнений}
	
	\begin{exampleblock}{Основная система уравнений}
		\parbox{\textwidth}{
			Будем исследовать систему квазилинейных дифференциальных уравнений от $n$ функций вида
			\begin{equation}
			\label{eq:main_eq}
			\vec{u}_t +  A\arguv \vec{u}_x = \vec{f}\arguv,
			\end{equation}
			где
			$\vec{u}\argtxo = \{ u_1\argtxo, u_2\argtxo, \ldots, u_n\argtxo \}^T$, 
			\[
			A\arguv =  \left(
			\begin{array}{cccc}
			a_{11}\arguv & a_{12}\arguv  & \ldots & a_{1n}\arguv\\
			a_{21}\arguv & a_{22}\arguv  & \ldots & a_{2n}\arguv\\
			\vdots          & \vdots           & \ddots & \vdots \\
			a_{n1}\arguv & a_{n2}\arguv  & \ldots & a_{nn}\arguv\\
			\end{array}
			\right),
			\]
			\[
			\vec{f}\arguv = \{ f_1\arguv, f_2\arguv, \ldots, f_n\arguv \}^T.
			\]
		}
	\end{exampleblock}
	

	
}

\frame{
	\frametitle{Характеристики системы квазилинейных уравнений }

	\begin{exampleblock}{Собственные числа и собственные векторы матрицы $A^T$}
	\parbox{\textwidth}{
		Пусть матрица $A^T\arguv$ имеет собственное число $\lambda\arguv$, которому соответствует собственный вектор $\vec{\alpha}\arguv$:
		\begin{equation}
		\label{eq:eigen_AT}
		A^T \vec{\alpha} = \lambda \vec{\alpha} \quad (\vec{\alpha} \neq 0).
		\end{equation}
		
	}
	\end{exampleblock}
	\begin{exampleblock}{Преобразования исходной системы}
	\parbox{\textwidth}{
		Умножим систему (\ref{eq:main_eq}) скалярно на вектор $\vec{\alpha}\arguv$ и преобразуем в соответствие с (\ref{eq:eigen_AT}), тогда
		\[
		\vec{u}_t\cdot\vec{\alpha} +  (A \vec{u}_x) \cdot\vec{\alpha} = \vec{f}\cdot\vec{\alpha}.
		\]
		Выражение преобразуется 
		\[
		(A \vec{u}_x) \cdot\vec{\alpha} = \vec{u}_x \cdot (A^T\vec{\alpha}) = \vec{u}_x \cdot\lambda \vec{\alpha} = (\lambda  \vec{u}_x) \cdot\vec{\alpha}.
		\]
		
%		\begin{equation}
%		\label{eq:main_eq_alpha}
%		\[
%		\vec{u}_t\cdot\vec{\alpha} +  \vec{u}_x \cdot (A^T\vec{\alpha}) = \vec{f}\cdot\vec{\alpha},
%		\]
%		\end{equation}
	}
	\end{exampleblock}
}

\frame{
	\frametitle{Характеристики системы квазилинейных уравнений}
	
	\begin{exampleblock}{Характеристическая форма записи}
		\parbox{\textwidth}{
		Основная система, записанная в форме
		\begin{equation}
		\label{eq:main_character}
		(\vec{u}_t + \lambda \vec{u}_x)	\cdot \vec{\alpha} = \vec{f}\cdot\vec{\alpha},
		\end{equation}
		называется \alert{характеристической формой}~$\lambda$.
		
		\medskip
		Если у матрицы $A^T$ имеется $n$ вещественных собственных чисел и полная система из $n$ линейно независимых собственных векторов, тогда всю систему (\ref{eq:main_eq}) можно переписать в  виде (\ref{eq:main_character}) и она будет называться \alert{гиперболической}. 
		}
	\end{exampleblock}
	
}

\frame{
	\frametitle{Инварианты Римана системы квазилинейных уравнений}
	\begin{exampleblock}{Инварианты Римана}
		\parbox{\textwidth}{
			Пусть $F\arguv$ является потенциалом для собственного вектора $\vec{\alpha}\arguv$ 
			\[
			\nabla_{u} F = \vec{\alpha},
			\]
			тогда $F(\vec{u})$ называют \alert{инвариантом Римана}.
			
		}
	\end{exampleblock}
}



\frame{
	\frametitle{Инварианты Римана системы квазилинейных уравнений}

	\parbox{\textwidth}{
		Рассмотрим кривую в плоскости $\argtxo$, называемую \alert{характеристической}, удовлетворяющую уравнению
		\begin{equation}
		\label{eq:eq_ch_dir}
		\od{x}{t} = \lambda(\vec{u}\argtxo),
		\end{equation}
		где $\vec{u} = \vec{u}\argtxo$ -- решение исходной системы уравнений (\ref{eq:main_eq}).
		
		\medskip
		Тогда полная производная от инварианта Римана $F(t, x(t))$ вдоль характеристической кривой (\ref{eq:eq_ch_dir}) имеет вид
		\[
		\od{F}{t} = \pd{F}{t} + \lambda \pd{F}{x} =  \nabla_{u} F \cdot \od{\vec{u}}{t} =  
		\alpha \cdot( \vec{u}_t  + \lambda \vec{u}_x ) = \vec{\alpha} \cdot \vec{f}.
		\]
		
		\medskip
		Если у системы (\ref{eq:main_eq}) имеется $n$ существенно различных инвариантов Римана, тогда она может быть проинтегрирована вдоль характеристик.
	
	}

	
}


\frame{
	\frametitle{Инварианты Римана для уравнений газовой динамики}
	
	\begin{exampleblock}{Одномерная система уравнений газовой динамики}
		\parbox{\textwidth}{
		
		\begin{columns}
			\begin{column}{0.5\textwidth}
				\begin{eqnarray*}
					\rho_t + v\rho_x + \rho v_x & = & 0,\\
					v_t + v v_x + \frac{p_x}{\rho} & = & 0,\\
					S_t + v S_x & = & 0.
				\end{eqnarray*}
			\end{column}
			\begin{column}{0.5\textwidth}
				Калорическое уравнение состояния
				\[
					p = p(\rho, S).
				\]
			\end{column}
		\end{columns}

				
		}
		\end{exampleblock}
		\pause
		
		\medskip	
		\begin{exampleblock}{Матричная форма записи}
			\parbox{\textwidth}{
					\[
			u_t + A u_x = 0,
			\]
			\[
			u = \left(
			\begin{array}{c}
			\rho\\ v\\ S
			\end{array}
			\right),\quad
			A = \left(
			\begin{array}{ccc}
			v & \rho & 0\\
			c^2/\rho & v & 0\\
			0 & 0 & v\\
			\end{array}
			\right).
			\]	
			}
		\end{exampleblock}


}


\frame{
	\frametitle{ Литература }
	\begin{itemize}[partopsep=1pt,label=\textbullet]
		\item 
%		\textbf{Л.И. Седов.} {\em Механика сплошной среды}. Том 2. М.:Наука, 1970.
		
		\item 
%		\textbf{Лойцянский~Л.~Г.} {\edocumentm  Механика жидкости газа и плазмы}. Учеб. для вузов. --- 7-е изд., испр. -- М.:Дрофа, 2003
	\end{itemize}
}

\end{document}
