\documentclass{beamer}

\usepackage{beamerthemesplit}
\usetheme{Singapore} %Copenhagen}
%\usecolortheme{whale}

\beamertemplatenavigationsymbolsempty % Hide navigation panel

%\usepackage[T2A]{fontenc}
%\usepackage[utf8]{inputenc}
%\usepackage[russian]{babel}

\usepackage[main=russian,english]{babel}   %% загружает пакет многоязыковой вёрстки
\usepackage{fontspec}      %% подготавливает загрузку шрифтов Open Type, True Type и др.
\defaultfontfeatures{Ligatures={TeX},Renderer=Basic}  %% свойства шрифтов по умолчанию
\setmainfont{Times New Roman} %% задаёт основной шрифт документа
%\usefonttheme{professionalfonts}% SOLUTION
\usefonttheme{serif}

\usepackage{hyperref}
\usepackage{textcomp}
\usepackage{amssymb,amsmath}
%\usepackage{animate}
%\usepackage{longtable}
\usepackage{xcolor}
\usepackage{icomma}

%\usepackage{pgffor}
\usepackage{enumitem}
\usepackage[export]{adjustbox}

\newcounter{N}

%% Форматирование окружения itemize
%\usepackage{ragged2e}
%\let\olditem\item
%\renewcommand\item{\olditem\justifying}

\usepackage{ mathrsfs }
\newcommand{\Rho}{\mathscr{P}}

\renewcommand{\Re}{\operatorname{Re}}
\newcommand{\Sh}{\operatorname{Sh}}
\newcommand{\Eu}{\operatorname{Eu}}
\newcommand{\Fr}{\operatorname{Fr}}

%\DeclareMathOperator{\tg}{tg}
\DeclareMathOperator{\сtg}{сtg}


\newcommand{\argxi}{(\xi^1,\xi^2,\xi^3)}
\newcommand{\argx}{(x^1,x^2,x^3)}

\newcommand{\argxiv}{(\vec{\xi})}
\newcommand{\argxv}{(\vec{x})}


\newcommand{\argxbarn}{(\bar{x}^1,\bar{x}^2,\ldots, \bar{x}^n)}
\newcommand{\argxn}{(x^1, x^2,\ldots, x^n)}

\newcommand{\argtxi}{(t, \xi^1,\xi^2,\xi^3)}
\newcommand{\argtoxi}{(t_0, \xi^1,\xi^2,\xi^3)}

\newcommand{\argtxiv}{(t, \vec{\xi})}
\newcommand{\argtoxiv}{(t_0, \vec{\xi})}

\newcommand{\argtxo}{(t, x)}
\newcommand{\arguv}{(\vec{u})}

\newcommand{\argtxv}{(t, \vec{x})}
\newcommand{\argtoxv}{(t_0, \vec{x})}


\newcommand{\pd}[2]{\frac{\partial #1}{\partial #2}}
\newcommand{\pdk}[2]{\frac{\partial^2 #1}{\partial #2^2}}

\newcommand{\od}[2]{\frac{d #1}{d #2}}
\newcommand{\odk}[3]{\frac{d^{#3} #1}{d #2^{#3}}}

\newcommand{\grad}{\operatorname{grad}}
\newcommand{\rot}{\operatorname{rot}}
\newcommand{\divo}{\operatorname{div}}

\newtheorem{dfn}{Определение}  
\newtheorem{theorems}{Теорема}  

\title[]{Решения со слабыми разрывами уравнений газовой динамики}

\author[]{ {\em Верещагин Антон Сергеевич}
\\
канд. физ.-мат. наук, старший преподаватель\\
\bigskip
Кафедра аэрофизики и газовой динамики ФФ НГУ}

\usebackgroundtemplate{\includegraphics[width=\paperwidth]{../img/background.png}}

\begin{document}
	
\frame{\titlepage}

\frame{
	\frametitle{Аннотация}
	\parbox{\textwidth}{

	}
}


\frame{
	\frametitle{Характеристики системы квазилинейных уравнений}
	
	\begin{exampleblock}{Основная система уравнений}
		\parbox{\textwidth}{
			Будем исследовать систему квазилинейных дифференциальных уравнений от $n$ функций вида
			\begin{equation}
			\label{eq:main_eq}
			\vec{u}_t +  A\arguv \vec{u}_x = \vec{f}\arguv,
			\end{equation}
			где
			$\vec{u}\argtxo = \{ u_1\argtxo, u_2\argtxo, \ldots, u_n\argtxo \}^T$, 
			\[
			A\arguv =  \left(
			\begin{array}{cccc}
			a_{11}\arguv & a_{12}\arguv  & \ldots & a_{1n}\arguv\\
			a_{21}\arguv & a_{22}\arguv  & \ldots & a_{2n}\arguv\\
			\vdots          & \vdots           & \ddots & \vdots \\
			a_{n1}\arguv & a_{n2}\arguv  & \ldots & a_{nn}\arguv\\
			\end{array}
			\right),
			\]
			\[
			\vec{f}\arguv = \{ f_1\arguv, f_2\arguv, \ldots, f_n\arguv \}^T.
			\]
		}
	\end{exampleblock}
	

	
}

\frame{
	\frametitle{Характеристики системы квазилинейных уравнений }

	\begin{exampleblock}{Собственные числа и собственные векторы матрицы $A^T$}
	\parbox{\textwidth}{
		Пусть матрица $A^T\arguv$ имеет собственное число $\lambda\arguv$, которому соответствует собственный вектор $\vec{\alpha}\arguv$:
		\begin{equation}
		\label{eq:eigen_AT}
		A^T \vec{\alpha} = \lambda \vec{\alpha} \quad (\vec{\alpha} \neq 0).
		\end{equation}
		
	}
	\end{exampleblock}
	\begin{exampleblock}{Преобразования исходной системы}
	\parbox{\textwidth}{
		Умножим систему (\ref{eq:main_eq}) скалярно на вектор $\vec{\alpha}\arguv$ и преобразуем в соответствие с (\ref{eq:eigen_AT}), тогда
		\[
		\vec{u}_t\cdot\vec{\alpha} +  (A \vec{u}_x) \cdot\vec{\alpha} = \vec{f}\cdot\vec{\alpha}.
		\]
		Выражение преобразуется 
		\[
		(A \vec{u}_x) \cdot\vec{\alpha} = \vec{u}_x \cdot (A^T\vec{\alpha}) = \vec{u}_x \cdot\lambda \vec{\alpha} = (\lambda  \vec{u}_x) \cdot\vec{\alpha}.
		\]
		
%		\begin{equation}
%		\label{eq:main_eq_alpha}
%		\[
%		\vec{u}_t\cdot\vec{\alpha} +  \vec{u}_x \cdot (A^T\vec{\alpha}) = \vec{f}\cdot\vec{\alpha},
%		\]
%		\end{equation}
	}
	\end{exampleblock}
}

\frame{
	\frametitle{Характеристики системы квазилинейных уравнений}
	
	\begin{exampleblock}{Характеристическая форма записи}
		\parbox{\textwidth}{
		Основная система, записанная в форме
		\begin{equation}
		\label{eq:main_character}
		(\vec{u}_t + \lambda \vec{u}_x)	\cdot \vec{\alpha} = \vec{f}\cdot\vec{\alpha},
		\end{equation}
		называется \alert{характеристической формой}~$\lambda$.
		
		\medskip
		Если у матрицы $A^T$ имеется $n$ вещественных собственных чисел и полная система из $n$ линейно независимых собственных векторов, тогда всю систему (\ref{eq:main_eq}) можно переписать в  виде (\ref{eq:main_character}) и она будет называться \alert{гиперболической}. 
		}
	\end{exampleblock}
	
}

\frame{
	\frametitle{Инварианты Римана системы квазилинейных уравнений}
	\begin{exampleblock}{Инварианты Римана}
		\parbox{\textwidth}{
			Пусть $F\arguv$ является потенциалом для собственного вектора $\vec{\alpha}\arguv$ 
			\[
			\nabla_{u} F = \vec{\alpha},
			\]
			тогда $F(\vec{u})$ называют \alert{инвариантом Римана}.
			
		}
	\end{exampleblock}
}



\frame{
	\frametitle{Инварианты Римана системы квазилинейных уравнений}

	\parbox{\textwidth}{
		Рассмотрим кривую в плоскости $\argtxo$, называемую \alert{характеристической}, удовлетворяющую уравнению
		\begin{equation}
		\label{eq:eq_ch_dir}
		\od{x}{t} = \lambda(\vec{u}\argtxo),
		\end{equation}
		где $\vec{u} = \vec{u}\argtxo$ -- решение исходной системы уравнений (\ref{eq:main_eq}).
		
		\medskip
		Тогда полная производная от инварианта Римана $F(t, x(t))$ вдоль характеристической кривой (\ref{eq:eq_ch_dir}) имеет вид
		\[
		\od{F}{t} = \pd{F}{t} + \lambda \pd{F}{x} =  \nabla_{u} F \cdot \od{\vec{u}}{t} =  
		\alpha \cdot( \vec{u}_t  + \lambda \vec{u}_x ) = \vec{\alpha} \cdot \vec{f}.
		\]
		
		\medskip
		Если у системы (\ref{eq:main_eq}) имеется $n$ существенно различных инвариантов Римана, тогда она может быть проинтегрирована вдоль характеристик.
	
	}

	
}


\frame{
	\frametitle{Характеристический вид уравнений газовой динамики}
	
	\begin{exampleblock}{Одномерная система уравнений газовой динамики}
		\parbox{\textwidth}{
		
		\begin{columns}
			\begin{column}{0.5\textwidth}
				\begin{eqnarray*}
					\rho_t + v\rho_x + \rho v_x & = & 0,\\
					v_t + v v_x + \frac{p_x}{\rho} & = & 0,\\
					S_t + v S_x & = & 0.
				\end{eqnarray*}
			\end{column}
			\begin{column}{0.5\textwidth}
				Калорическое уравнение состояния
				\[
					p = p(\rho, S).
				\]

			\end{column}
		\end{columns}

				
		}
		\end{exampleblock}
		\pause
		
		\medskip	
		\begin{exampleblock}{Матричная форма записи}
			\parbox{\textwidth}{
					\[
			u_t + A u_x = 0,
			\]
			\[
			u = \left(
			\begin{array}{c}
			\rho\\ v\\ S
			\end{array}
			\right),\quad
			A = \left(
			\begin{array}{ccc}
			v & \rho & 0\\
			\displaystyle c^2/\rho & v & p_S/\rho\\
			0 & 0 & v\\
			\end{array}
			\right),\quad
			c^2 = \pd{p}{\rho}(\rho,S).
			\]	
			}
		\end{exampleblock}


}

\frame{
	\frametitle{Характеристический вид уравнений газовой динамики }
	
	\begin{exampleblock}{Характеристическое уравнение}
		\parbox{\textwidth}{
			\[
			\chi(\lambda) = (v - \lambda)((v-\lambda)^2 - c^2)=0\iff
			\lambda_{1,2} = v \pm c, \quad\lambda_3 = v.
			\]
		}
	\end{exampleblock}	
	\begin{exampleblock}{Собственные векторы}
		\parbox{\textwidth}{
			\[
			\lambda_1 = v-c \Rightarrow \alpha_3 = \left(-\frac{c}{\rho}, 1, -\frac{1}{\rho c}p_S\right).
			\]
			\[
			\lambda_2 = v+c \Rightarrow \alpha_2 = \left(\frac{c}{\rho}, 1, \frac{1}{\rho c}p_S\right),			
			\]
			\[
			\lambda_3 = 0 \Rightarrow \alpha_1 = (0, 0, 1),			
			\]

			
		}
	\end{exampleblock}
	
	
}

\frame{
	\frametitle{Характеристический вид уравнений газовой динамики }
	
	\begin{exampleblock}{Запись через частные производные}
		\parbox{\textwidth}{
		\[
		S_t + v S_x = 0,
		\]
		\[
		v_t + (v-c)v_x -
		\frac{c}{\rho} \left[\rho_t + (v-c)\rho_x	\right]-
		\frac{1}{\rho c} \pd{p}{S}\left[ S_t + (v-c) S_x \right] = 0,
		\]
		\[
		v_t + (v+c)v_x +
		\frac{c}{\rho} \left[\rho_t + (v+c)\rho_x	\right] +
		\frac{1}{\rho c} \pd{p}{S}\left[ S_t + (v+c) S_x \right] = 0,
		\]
			
		}
	\end{exampleblock}
	
	
	\begin{exampleblock}{Запись в дифференциалах}
		\parbox{\textwidth}{
		\[
		dx = (v-c) dt, \quad dv - \frac{c}{\rho} d\rho - \pd{p}{S}\frac{1}{\rho c} dS = 0,
		\]
		\[
		dx = v dt,\quad dS  = 0,
		\]
		\[
		dx = (v+c) dt, \quad dv + \frac{c}{\rho} d\rho + \pd{p}{S}\frac{1}{\rho c} dS = 0.
		\]
		}
	\end{exampleblock}
	
}

\frame{
	\frametitle{ Инварианты Римана для изоэнтропических течений}
	
	\begin{exampleblock}{Условия}
		\parbox{\textwidth}{
			Пусть $S\argtxo = S_0$ в всей области течения, тогда 
			\[
			p = p(\rho, S_0) \Rightarrow \pd{p}{S} = 0, \quad c(\rho) = \sqrt{\left(\pd{p}{\rho}\right)_S}.
			\]
			
		}
	\end{exampleblock}
	\begin{exampleblock}{Инварианты Римана}
		\parbox{\textwidth}{

			\only<1>{				
			Найдём $s(\rho,v)$ и $r(\rho,v)$ такие, что 
			$
			\nabla_{(\rho,v)} s = \alpha_1$, $\nabla_{(\rho,v)} r = \alpha_2$.
			\[
			\pd{s}{\rho} = -\frac{c(\rho)}{\rho},\quad
			\pd{s}{v} = 1 \Rightarrow
			s = v - \int\frac{c(\rho)}{\rho} d\rho.
			\]
			\[
			\pd{r}{\rho} = \frac{c(\rho)}{\rho},\quad
			\pd{r}{v} = 1 \Rightarrow
			r = v + \int\frac{c(\rho)}{\rho} d\rho.
			\]

			}
		
			\only<2>{
			\[
			\pd{s}{t} + (v-c) \pd{s}{x} = 0,\quad \pd{r}{t} + (v+c) \pd{r}{x} = 0.
			\]
			Полученные $s(\rho,v)$ и $r(\rho,v)$ называются левым и правым инвариантом Римана соответственно.
			
			}
			
			
		}
	\end{exampleblock}
	
}


\frame{
	\frametitle{Инварианты Римана для политропного газа}
	
	\begin{exampleblock}{Уравнение состояния}
		\parbox{\textwidth}{
			\[
			p = a(S) \rho^\gamma,\quad \gamma = \frac{c_p}{c_V} > 1,
			\]
			\[
			c^2 = a(S)\gamma \rho^{\gamma-1} = \frac{\gamma p}{\rho}.
			\]
			
		}
	\end{exampleblock}
	\begin{exampleblock}{Инварианты Римана}
		\parbox{\textwidth}{
			\[
			s = v - \int \frac{c(\rho)}{\rho} d\rho = 
			v- \frac{2}{\gamma-1}c,\quad
			r = v + \int \frac{c(\rho)}{\rho} d\rho = 
			v + \frac{2}{\gamma-1}c.
			\]
			\[\Downarrow\]
			\[
			v = \frac{r+s}{2},\quad
			c = \frac{\gamma-1}{4}(r-s).
			\]
			
		}
	\end{exampleblock}
	
}

\frame{
	\frametitle{Система дифференциальных уравнений изоэнтропического течения политропного газа}

	\begin{exampleblock}{Общий случай}
		\parbox{\textwidth}{
		\[
			\pd{s}{t}+(\alpha s + \beta r)\pd{s}{x} = 0,\quad
			\pd{r}{t}+(\alpha r + \beta s)\pd{s}{x} = 0,
		\]
		где
		\[
		\alpha = \frac{1}{2}+\frac{\gamma-1}{4}>\frac{1}{2}>0,\quad
		\beta  = \frac{1}{2}-\frac{\gamma-1}{4}.
		\]
		}
	\end{exampleblock}\pause
	
	\begin{exampleblock}{Газ Чаплыгина $\gamma=1$}
		\parbox{\textwidth}{
		\[
			\pd{s}{t}+r\pd{s}{x} = 0,\quad
			\pd{r}{t}+s\pd{s}{x} = 0.
		\]	
		}
	\end{exampleblock}\pause

	\begin{exampleblock}{Случай $\gamma=3$}
		\parbox{\textwidth}{
		\[		
			\pd{s}{t}+s\pd{s}{x} = 0,\quad
			\pd{r}{t}+r\pd{s}{x} = 0.
		\]
		}			
	\end{exampleblock}
}

\frame{
	\frametitle{Задача Коши для изоэнтропического течения политропного газа}
	
	\begin{exampleblock}{Общий случай}
		\parbox{\textwidth}{
			\[
			\pd{s}{t}+(\alpha s + \beta r)\pd{s}{x} = 0,\quad
			\pd{r}{t}+(\alpha r + \beta s)\pd{s}{x} = 0,
			\]
			где
			\[
			\alpha = \frac{1}{2}+\frac{\gamma-1}{4},\quad
			\beta  = \frac{1}{2}-\frac{\gamma-1}{4},\quad
			\gamma \neq 1.
			\]
			Начальные условия при $t=0$:
			\[
			s(x,0) = s_0(x),\quad
			r(x,0) = r_0(x).
			\]
			
			В этом случае, из общей теории, следует существование решения в некоторой полосе $0 \leq t < t_0$; величина $t_0$ есть момент времени, в который производные решения становятся неограниченными.
		}
	\end{exampleblock}
	
}


\frame{
	\frametitle{Бегущие волны (волны Римана)}
	
	\begin{exampleblock}{Определение}
		\parbox{\textwidth}{
			Если в какой-то области изоэнтропического течения один из инвариантов Римана остаётся постоянным, то такое течение называют \alert{волной Римана} или \alert{бегущей волной}. 
		}
	\end{exampleblock}


	
}

\frame{
	\frametitle{Характеристики в области бегущей волны}
	
	\begin{exampleblock}{Уравнения бегущей волны}
		\parbox{\textwidth}{
			Пусть в некоторой области $r = r_0 = const$, тогда течение будет описываться уравнением
			\[
			\pd{s}{t}+(\alpha s + \beta r_0)\pd{s}{x} = 0,
			\]	
			где
			\[
			\alpha = \frac{1}{2}+\frac{\gamma-1}{4},\quad
			\beta  = \frac{1}{2}-\frac{\gamma-1}{4},\quad
			\gamma \neq 1.
			\]
			
		}
	\end{exampleblock}\pause
	\begin{exampleblock}{Уравнения характеристик}
		\parbox{\textwidth}{
			Вдоль линии 
			\[
			\od{x}{t} = \frac{x-x_0}{t-t_0} = \alpha s(x,t) + \beta r_0
			\]
			сохраняется инвариант $s(x,t)$, это означает, что характеристики будут \alert{прямыми линиями}.
		}
	\end{exampleblock}
	
}

\frame{
	\frametitle{Волны сжатия и разрежения в случае бегущей $r$-волны}
	
	\begin{exampleblock}{Цепочка алгебраических следствий}
		\parbox{\textwidth}{
			\[
			\pd{s}{x} > 0 \quad\text{и}\quad
			\begin{array}{lcccc}
			v = \displaystyle\frac{1}{2}(r_0+s) & \Rightarrow & \displaystyle\pd{v}{x} > 0, & & \\
			c = \displaystyle\frac{\gamma-1}{4}(r_0-s) & \Rightarrow & \displaystyle\pd{c}{x} < 0 & \Rightarrow & \displaystyle\pd{\rho}{x} < 0.
			\end{array} 
			\]
		}
	\end{exampleblock}	\pause
	
	\begin{exampleblock}{Условия существования волн сжатия и разрежения}
		\parbox{\textwidth}{
		
			
			Таким образом, в области, где инвариант Римана $s\argtxo$ увеличивается, там происходит разгон течения с одновременным его \textit{расширением}. Такое течение будет называться \alert{волной разрежения}.  \pause
			
			\medskip
			В случае \textit{уменьшения} инварианта $s\argtxo$, наоборот, скорость $v\argtxo$ будет  \textit{уменьшаться}, а плотность $\rho\argtxo$  \textit{увеличиваться}. Такая бегущая волна будет называться \alert{волной сжатия}.
			
		}
	\end{exampleblock}
}

\frame{
	\frametitle{Волны сжатия и разрежения в случае бегущей $s$-волны}
	
	\begin{exampleblock}{Цепочка алгебраических следствий}
		\parbox{\textwidth}{
			\[
			\pd{r}{x} > 0 \quad\text{и}\quad
			\begin{array}{lcccc}
			v = \displaystyle\frac{1}{2}(r+s_0) & \Rightarrow & \displaystyle\pd{v}{x} > 0, & & \\
			c = \displaystyle\frac{\gamma-1}{4}(r-s_0) & \Rightarrow & \displaystyle\pd{c}{x} > 0 & \Rightarrow & \displaystyle\pd{\rho}{x} > 0.
			\end{array} 
			\]
		
		}
	\end{exampleblock}		\pause	
	\begin{exampleblock}{Условия существования волн сжатия и разрежения}
		\parbox{\textwidth}{
			
			
			Таким образом, в области, где инвариант Римана $r\argtxo$ \textit{увеличивается} будет реализовываться \alert{волна сжатия}.  \pause
			
			\medskip
			И наоборот, в случае \textit{уменьшения} инварианта $s\argtxo$ будет реализовываться \alert{волна разрежения}.
			
		}
	\end{exampleblock}
}

\frame{
	\frametitle{Центрированные волны}
	
	\begin{exampleblock}{Определение}
		\parbox{\textwidth}{
			Волна Римана ($r=r_0$) называется \alert{центрированной}, если $s$-характеристики образуют пучок прямых, выходящих из одной точки $(t_0,x_0)$.
			Так как, $s$ постоянен вдоль любой характеристики, то 		
			\[
			s = s\left(\frac{x-x_0}{t-t_0} \right),\quad r=r_0.
			\]			
		}
	\end{exampleblock}

	\begin{exampleblock}{Определение}
	\parbox{\textwidth}{
		Волна Римана ($s=s_0$) называется \alert{центрированной}, если $r$-характеристики образуют пучок прямых, выходящих из одной точки $(t_0,x_0)$.
				Так как, $r$ постоянен вдоль любой характеристики, то 
		\[
		r = r\left(\frac{x-x_0}{t-t_0} \right),\quad s=s_0.
		\]			
	}
	\end{exampleblock}
	
}


\frame{
	\frametitle{ Литература }
	\begin{itemize}[partopsep=1pt,label=\textbullet]
		\item 
%		\textbf{Л.И. Седов.} {\em Механика сплошной среды}. Том 2. М.:Наука, 1970.
		
		\item 
%		\textbf{Лойцянский~Л.~Г.} {\edocumentm  Механика жидкости газа и плазмы}. Учеб. для вузов. --- 7-е изд., испр. -- М.:Дрофа, 2003
	\end{itemize}
}

\end{document}
