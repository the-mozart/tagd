\documentclass{beamer}

\usepackage{beamerthemesplit}
\usetheme{Singapore} %Copenhagen}

\input{../../include/preamble.inc} 
\input{../../include/definitions.inc} 
\input{../../include/author.inc} 

\title[]{Элементы тензорного исчисления}

\usebackgroundtemplate{\includegraphics[width=\paperwidth]{../img/background.png}}

\begin{document}

\begin{frame}[plain]	
\titlepage
\end{frame}


\frame[plain]{
	\frametitle{Аннотация}
	\parbox{\textwidth}{
		Криволинейные системы координат. Скаляр. Вектор. Ковариантность и контравариантность. Тензор. Тензорная алгебра. Произведение тензоров. Сокращение индексов. Теоремы о тензорах. Фундаментальная квадратичная форма и тензор. Метрика. Скалярное произведение векторов.
	}
}


\frame{
	\frametitle{ Системы координат }
	
	\begin{exampleblock}{Определение}
	\parbox{\textwidth}{
		

		
	Пусть задана связь между криволинейными системами координат
	\begin{equation}
	\label{eq:coord_tranformation}
	x^\alpha = x^\alpha \argxbarn, \quad (\alpha = 1,2,\ldots,n)
	\end{equation}
	и
	\[
	\frac{D\argxn}{D\argxbarn}=
	\begin{vmatrix}
	\pd{x^1}{\bar{x}^1} &   \pd{x^1}{\bar{x}^2} &    \ldots &     \pd{x^1}{\bar{x}^n}   \\
	\pd{x^2}{\bar{x}^1} &   \pd{x^2}{\bar{x}^2} &    \ldots &     \pd{x^2}{\bar{x}^n}   \\
	\vdots & \vdots & \ddots & \vdots \\
	\pd{x^n}{\bar{x}^1} &   \pd{x^n}{\bar{x}^2} &    \ldots &     \pd{x^n}{\bar{x}^n}   
	\end{vmatrix}
	\neq 0,
	\]
	тогда существует обратное преобразование координат:
	\[
	\bar{x}^\alpha = \bar{x}^\alpha \argxn, \quad (\alpha = 1,2,\ldots,n).
	\]

	}		
	\end{exampleblock}
	
}


\frame{
	\frametitle{ Скаляр }
	
	\begin{exampleblock}{Определение}
		\parbox{\textwidth}{
			
			Если для каждой системы координат $\argxn$ определена функция $f\argxn$ такая, что при преобразовании системы координат   (\ref{eq:coord_tranformation}) выполняется условие
			\[
			f\argxn = f\argxbarn,
			\]
			то говорят, что функция точек $f\argxn$ есть \alert{инвариант}, или \alert{скаляр}.
			
		}
		
	\end{exampleblock}


	
}

\frame{
	\frametitle{ Контравариантный вектор }
	
	\begin{exampleblock}{Определение}
		\parbox{\textwidth}{
			
		Если для каждой системы координат $\argxn$ определена совокупность $n$ функций $A^1$, $A^2$,\ldots, $A^n$ такая, что для системы координат $\argxbarn$  мы имеем свою совокупность функций $\bar{A}^1$, $\bar{A}^2$,\ldots, $\bar{A}^n$, и если при преобразовании координат (\ref{eq:coord_tranformation}) эти функции преобразуются по следующим формулам:
		\[
		\bar{A}^i=\sum\limits_{\alpha=1}^n\pd{\bar{x}^i}{x^\alpha}A^\alpha \quad
		(i=1,\ldots,n),
		\]
		то говорят, что совокупность величин $A^1$, $A^2$,\ldots, $A^n$ определяет \alert{контравариантный вектор}, а величины $A^i$ называются \alert{компонентами контравариантного вектора} $A^i$.
			 
			
		}
		
	\end{exampleblock}
	
}

\frame{
	\frametitle{ Пример контравариантного вектора}

	\begin{exampleblock}{Соглашение}
		\medskip
		\parbox{\textwidth}{
		Условимся проводить суммирование по повторяющимся индексам в одночлене от $1$ до $n$, если не сделано отдельных оговорок заранее.
		}
		
	\end{exampleblock}

	\pause
	
	\begin{exampleblock}{Вектор $d\bar{x}^i$}
	\[
	d\bar{x}^i = \pd{\bar{x}^i}{x^1}dx^1+\pd{\bar{x}^i}{x^2}dx^2+\ldots+
	\pd{\bar{x}^i}{x^n}dx^n = \pd{\bar{x}^i}{x^\alpha}dx^\alpha
	\]

	\end{exampleblock}
	
	
}

\frame{
	\frametitle{ Пример ковариантного вектора }
	
	
	\begin{exampleblock}{Градиент функции}
		
		Рассмотрим градиент функции $\nabla \varphi\argxn$:
		\[
			\pd{\varphi}{\bar{x}^1},\quad
			\pd{\varphi}{\bar{x}^2},\quad
			\ldots,\quad
			\pd{\varphi}{\bar{x}^n}.
		\]
			\pause
		По правилам дифференцирования:
		\[
		\pd{\varphi}{\bar{x}^i} = 
		\pd{\varphi}{x^1}\pd{x^1}{\bar{x}^i}+
		\pd{\varphi}{x^2}\pd{x^2}{\bar{x}^i}+
		\ldots+
		\pd{\varphi}{x^n}\pd{x^n}{\bar{x}^i}=
		\pd{\varphi}{x^\alpha}\pd{x^\alpha}{\bar{x}^i}.
		\]
		
		
	\end{exampleblock}
	
	\pause

	\parbox{\textwidth}{
	Положим, что
	\[
	\pd{\varphi}{x^\alpha}	= A_\alpha,\quad
	\pd{\varphi}{\bar{x}^\alpha} = \bar{A}_\alpha,
	\]	
	тогда получим
	\alert{
	\[
	\bar{A}_i = A_\alpha \pd{x^\alpha}{\bar{x}^i}.
	\]
	}
	}
	
}


\frame{
	\frametitle{ Ковариантный вектор }
	
	\begin{exampleblock}{Определение}
	\parbox{\textwidth}{
		Если для каждой системы координат $x^\alpha$ определена совокупность $n$ функций $A_\alpha$ и если при преобразовании координат (\ref{eq:coord_tranformation}) эти функции преобразуются по формуле 
		\[
		\bar{A}_i = A_\alpha \pd{x^\alpha}{\bar{x}^i},
		\]
		то величины $A_\alpha$ определяют \alert{ковариантный вектор}, составляющими, или компонентами, которого они являются.
	}
	
		
	\end{exampleblock}
	\begin{exampleblock}{Соглашение}
		\smallskip
		\parbox{\textwidth}{
			
			Будем различать ковариантные векторы от контравариантных тем, что у контравариантных векторов индексы будем ставить сверху (например, $dx^i$), а у ковариантных -- снизу (например, $A_j$).
			
		}
		
	\end{exampleblock}
}

\frame{
	\frametitle{ Тензор второго ранга (контравариантный)}
	\begin{exampleblock}{Определение}
		\parbox{\textwidth}{
			Если для каждой системы координат $x^\alpha$ определена совокупность $n^2$ функций $A^{\alpha\beta}$, которые при преобразовании координат (\ref{eq:coord_tranformation}) испытывают преобразование
			\[
				\bar{A}^{ik} = A^{\alpha\beta}\pd{\bar{x}^i}{x^\alpha}\pd{\bar{x}^k}{x^\beta},
			\]
			то эти функции определяют \alert{контравариантный тензор второго ранга}, составляющими которого они являются.
			

		}
		
	\end{exampleblock}
	
}







\frame{
	\frametitle{Тензор второго ранга (ковариантный) }

\begin{exampleblock}{Определение}

	Если же определена совокупность $n^2$ функций $A_{\alpha\beta}$, которые при преобразовании координат (\ref{eq:coord_tranformation}) испытывают преобразование
	\[
	\bar{A}_{ik} = A_{\alpha\beta}\pd{x^\alpha}{\bar{x}^i}\pd{x^\beta}{\bar{x}^k},
	\]
	то эти функции определяют \alert{ковариантный тензор второго ранга}, составляющими которого они являются.
	
\end{exampleblock}
	
}
	
	
	\frame{
		\frametitle{Тензор второго ранга (смешанный) }
		
		\begin{exampleblock}{Определение}
			
			Если же определена совокупность $n^2$ функций $A_{\alpha\cdot}^{\cdot\beta}$, которые при преобразовании координат (\ref{eq:coord_tranformation}) испытывают преобразование
			\[
			\bar{A}_{i\cdot}^{\cdot k} = A_{\alpha\cdot}^{\cdot\beta}\pd{x^\alpha}{\bar{x}^i}\pd{\bar{x}^k}{x^\beta},
			\]
			то эти функции определяют \alert{смешанный тензор второго ранга}, составляющими которого они являются.
			
		\end{exampleblock}
		
 	}
 
 \frame{
 	\frametitle{ Пример смешанного тензора }
 	
 	\begin{exampleblock}{Символ Кронекера}
 		\[
 			\delta_{\alpha}^{\beta}= \left\{
 			\begin{array}{cc}
 			1, \quad \alpha=\beta, \\
 			0, \quad \alpha\neq\beta
 			\end{array}
 			\right.
 		\]
 		является смешанным тензором 2-го ранга.
 	\end{exampleblock}
 
 	
 
 	\begin{exampleblock}{Доказательство}
 	

 	\parbox{\textwidth}{
 		\[
 			\bar{\delta}_i^k=\delta_{\alpha}^{\beta} \pd{x^\alpha}{\bar{x}^i} \pd{\bar{x}^k}{x^\beta}=
 			\pd{x^\alpha}{\bar{x}^i} \pd{\bar{x}^k}{x^\alpha} =
 			\left\{
 			\begin{array}{cc}
 			1, \quad i=k, \\
 			0, \quad i \neq k.
 			\end{array}
 			\right.
 		\]
	}
 
	 \end{exampleblock}		
 
 
 	
 }
 
 
 \frame{
 	\frametitle{ Тензоры более высоких рангов }
 	
 	\begin{exampleblock}{Пример}
 		
 		\parbox{\textwidth}{
 			По аналогии можно ввести тензоры более высоких рангов -- это такой набор величин, для которых при переходе из одной системы координат в другую (\ref{eq:coord_tranformation}) выполняются аналогичные соотношения, например:
 			\[
 				\bar{A}_{ik}^l = A_{\alpha\beta}^{\gamma} \pd{x^\alpha}{\bar{x}^i} \pd{x^\beta}{\bar{x}^k} \pd{\bar{x}^l}{x^\gamma}.
 			\]
 			
 			В данном случае тензор $A_{\alpha\beta}^{\gamma}$ является \alert{смешанным тензором 3-го ранга} дважды ковариантным и один раз контравариантным.
 		}
 		
 	\end{exampleblock}
 }
 
 
 \frame{
 	\frametitle{ Тензорная алгебра: умножение на число и сумма }
 	
 	\begin{exampleblock}{Умножение тензора на число}
 		\parbox{\textwidth}{
 		При \alert{умножении} всех компонентов тензора $A_{\alpha\beta}^{\gamma}$ \alert{на число} $\lambda$ получается новый тензор с компонентами $\lambda A_{\alpha\beta}^{\gamma}$ (док-во очевидно).
		} 		
 	\end{exampleblock}
 	
	\pause
 	
 	\begin{exampleblock}{Сложение}
		\parbox{\textwidth}{
			\alert{Суммой} двух тензоров $A_{\alpha\beta}^{\gamma}$ и $B_{\alpha\beta}^{\gamma}$ одинаковой размерности будет тензор $C_{\alpha\beta}^{\gamma}$ такого же вида с компонентами
			\[
			C_{\alpha\beta}^{\gamma} = A_{\alpha\beta}^{\gamma} + B_{\alpha\beta}^{\gamma}.
			\]
			
		} 		
 	\end{exampleblock}
 
 	\parbox{\textwidth}{
 	Видно, что сложение тензоров обладает свойствами \alert{коммутативности}, \alert{ассоциативности}, а вместе с операцией умножение на число образует \alert{векторное пространство}.	
 	}
 	
 }
 
 \frame{
 	\frametitle{Тензорная алгебра: симметричность, антисимметричность }
 	\begin{exampleblock}{Симметричность}
 	\parbox{\textwidth}{
 		Тензор $A^{\alpha\beta}$ называется \alert{симметричным}, если для всех $\alpha$ и $\beta$
 		\[
 		A^{\alpha\beta}=A^{\beta\alpha}.
 		\]
 	}
 		
 	\end{exampleblock}
 
 	\pause
 
  	\begin{exampleblock}{Антисимметричность}
 	\parbox{\textwidth}{
 		Тензор $A^{\alpha\beta}$ называется \alert{антисимметричным}, если для всех $\alpha$ и $\beta$
 		\[
 		A^{\alpha\beta}=-A^{\beta\alpha}.
 		\]
 	}
 	
	\end{exampleblock}

	\pause
	
	\begin{exampleblock}{}
		\parbox{\textwidth}{
			По аналогии с ортогональными тензорами 2-го ранга можно показать, что любой тензор можно представить в виде суммы симметричного и антисимметричного по двум выбранным (одновременно ковариантным или контравариантным) индексам, причем единственным образом. 			
		}
		
	\end{exampleblock}
 }
 
 \frame{
 	\frametitle{ Тензорная алгебра: произведение тензоров }
 	
 	\begin{exampleblock}{Определение}
 		\parbox{\textwidth}{
 			
 		Назовем произведением тензора $A_{\alpha}^\beta$ и тензора $B_{\gamma\delta}^\varepsilon$ тензор $C_{\alpha\gamma\delta}^{\beta\varepsilon}$, образуемый по формуле
    	$C_{\alpha\gamma\delta}^{\beta\varepsilon}=A_{\alpha}^\beta B_{\gamma\delta}^\varepsilon$.
 		}		
 	\end{exampleblock}
 	\pause
  	\begin{exampleblock}{Корректность определения}
 	\parbox{\textwidth}{
 		Для тензоров $A_{\alpha}^\beta$ и $B_{\gamma\delta}^\varepsilon$  справедливы соотношения:
 		\[
 			\bar{A}_i^k = A_{\alpha}^\beta \pd{x^\alpha}{\bar{x}^i} \pd{\bar{x}^k}{x^\beta},\quad
 			\bar{B}_{lm}^n = B_{\gamma\delta}^\varepsilon \pd{x^\gamma}{\bar{x}^l} 
 			\pd{x^\delta}{\bar{x}^m} \pd{\bar{x}^n}{x^\varepsilon},
 		\]
 		следовательно компоненты $C_{\alpha\gamma\delta}^{\beta\varepsilon}$ в новой системе координат:
 		\[
 		\bar{C}_{ilm}^{kn} = \bar{A}_i^k \bar{B}_{lm}^n = C_{\alpha\gamma\delta}^{\beta\varepsilon} \pd{x^\alpha}{\bar{x}^i} \pd{\bar{x}^k}{x^\beta} \pd{x^\gamma}{\bar{x}^l} 
 		\pd{x^\delta}{\bar{x}^m} \pd{\bar{x}^n}{x^\varepsilon}.
 		\]

 	}		
	 \end{exampleblock}
 
 }
 
 \frame{
 	\frametitle{ Тензорная алгебра: сокращение индексов }
 	
 	\parbox{\textwidth}{
 		

 	
 	Докажем, что тензор $B_\alpha = A_{\alpha\beta}^{\beta}$, полученный из смешанного тензора $A_{\alpha\beta}^{\gamma}$ путем суммирования по повторяющемуся индексу $\beta$ от $1$ до $n$, является тензором 1-го ранга.
 	
 	
 	\medskip
 	Имеем
 	\[
 		\bar{A}_{ik}^l = A_{\alpha\beta}^{\gamma} \pd{x^\alpha}{\bar{x}^i} \pd{x^\beta}{\bar{x}^k} \pd{\bar{x}^l}{x^\gamma}, 
 	\]
 	тогда
 	\[
 		 \bar{B}_i = \bar{A}_{ik}^k = A_{\alpha\beta}^{\gamma} \pd{x^\alpha}{\bar{x}^i} \pd{x^\beta}{\bar{x}^k} \pd{\bar{x}^k}{x^\gamma} = 
 		A_{\alpha\beta}^{\gamma} \pd{x^\alpha}{\bar{x}^i} \delta_\gamma^\beta=
 		A_{\alpha\beta}^{\beta} \pd{x^\alpha}{\bar{x}^i} = B_\alpha \pd{x^\alpha}{\bar{x}^i},
 	\]
 	что доказывает, что $B_\alpha$ -- ковариантный вектор.
 	
	\begin{exampleblock}{Вывод}
		\parbox{\textwidth}{
			Для каждого смешанного тензора можно получить новый тензор путем \alert{сокращения} одного ковариантного и контравариантного \alert{индекса}. 
		}
	\end{exampleblock} 	
 	}
 }

\frame{
	\frametitle{ Примеры }
	\parbox{\textwidth}{
	\begin{itemize}
		\item Скалярное произведение векторов: $a = A^\alpha B_\alpha$. \pause
		\item Скалярное произведение тензора и вектора: $C_\beta = A_{\alpha\beta}B^\alpha$. \pause
		\item Бискалярное произведение тензоров: $b = A_{\alpha\beta}B^{\beta\alpha}$.
	\end{itemize}	
	}
	
	
}
 
 \frame{
 	\frametitle{ Теорема о тензорной природе объекта }
 	
 	\begin{exampleblock}{Теорема}
 		\parbox{\textwidth}{
 			Если для каждой системы координат $\argxn$ имеем совокупность $n^3$ величин $A_{\alpha\beta}^\gamma$ и если при любом выборе трех векторов $u^\alpha$, $v^\beta$ и $w_\gamma$ выражение 
 			\[
 				f = A_{\alpha\beta}^\gamma u^\alpha v^\beta w_\gamma
 			\]
 			является инвариантом, то величины $A_{\alpha\beta}^\gamma$ являются составляющими тензора два раза ковариантного один раз контравариантного.
 		}
 	\end{exampleblock}
 	
 }

\frame{
	\frametitle{Теорема о тензорной природе объекта }
	\begin{exampleblock}{Доказательство}
		\parbox{\textwidth}{
			В силу произвольности $u^\alpha$, $v^\beta$ и $w_\gamma$ возьмем их так, что для заранее заданных $i$, $j$, $k$ в новой системе координат: 
			\[
			\bar{u}^\alpha = \delta^\alpha_i,\quad
			\bar{v}^\beta = \delta^\beta_k,\quad
			\bar{w}_\gamma = \delta_\gamma^j.
			\] \pause
			Тогда 
			$\displaystyle
			u^\alpha = \pd{x^\alpha}{\bar{x}^r}\bar{u}^r=\pd{x^\alpha}{\bar{x}^r}\delta^r_i=\pd{x^\alpha}{\bar{x}^i}$, $\displaystyle v^\beta = \pd{x^\beta}{\bar{x}^k}$, $\displaystyle w_\gamma = \pd{\bar{x}^j}{x^\gamma}$. \pause
			\[
				\bar{f} = \bar{A}_{\alpha\beta}^\gamma \bar{u}^\alpha \bar{v}^\beta \bar{w}_\gamma =  \bar{A}^j_{ik},\quad \pause
				f = A_{\alpha\beta}^\gamma u^\alpha v^\beta w_\gamma =  A_{\alpha\beta}^\gamma \pd{x^\alpha}{\bar{x}^i} \pd{x^\beta}{\bar{x}^k} \pd{\bar{x}^j}{x^\gamma}.
			\]
			\pause
			По условию теоремы $f$ -- скаляр, значит, $\bar{A}_{\alpha\beta}^\gamma = A_{\alpha\beta}^\gamma \displaystyle\pd{x^\alpha}{\bar{x}^i} \pd{x^\beta}{\bar{x}^k} \pd{\bar{x}^j}{x^\gamma}$.
			
			
		}
	\end{exampleblock}
}

 \frame{
 	\frametitle{ Теорема о тензорной природе объекта 2 } 
 	
 	\begin{exampleblock}{Теорема}
 		\parbox{\textwidth}{
 			Если для каждой системы координат $x^\alpha$ имеется совокупность $n^2$ величин $A_{\alpha\beta}$ и если при любом выборе вектора $u^\alpha$ выражение
 			\[
 				f = A_{\alpha\beta} u^\alpha u^\beta
 			\]
 			является инвариантом, то величина 
 			\[
 			B_{\alpha\beta} = \frac{1}{2}\left( A_{\alpha\beta} + A_{\beta\alpha} \right)
 			\]
 			является составляющей ковариантного тензора. 
 			
 		}
 	\end{exampleblock}
  }
 
 
 \frame{
 	\frametitle{ Теорема о тензорной природе объекта 2 }
	 \begin{exampleblock}{Доказательство}
	 	\parbox{\textwidth}{
			Положим, $u^\alpha=v^\alpha + w^\alpha$ -- произвольное разложение вектора $u^\alpha$ в сумму двух векторов, тогда
			\[
			f = A_{\alpha\beta}(v^\alpha+w^\alpha)(v^\beta+w^\beta)= A_{\alpha\beta} v^\alpha v^\beta + A_{\alpha\beta} w^\alpha w^\beta + 
			\]
			\[
			+ A_{\alpha\beta} v^\alpha w^\beta + A_{\alpha\beta} v^\beta w^\alpha .
			\]
			Заметим, что 
			\[
			A_{\alpha\beta} v^\alpha v^\beta,\quad
			A_{\alpha\beta} w^\alpha w^\beta
			\]
			являются инвариантами по условию теоремы, а
			\[
				A_{\alpha\beta} v^\alpha w^\beta = A_{\beta\alpha} v^\alpha w^\beta.
			\]
			Поэтому выражение $g = (A_{\alpha\beta} + A_{\beta\alpha}) v^\alpha w^\beta$
			является инвариантом, а $A_{\alpha\beta}+A_{\beta\alpha}$ -- ковариантным тензором по теореме 1.
	 		
	 	}
	 \end{exampleblock}
 	
 }

\frame{
	\frametitle{ Признак тензорной природы для симметричного объекта }
	
	\begin{exampleblock}{Следствие}
		\smallskip
		\parbox{\textwidth}{
			Если величины $A_{\alpha\beta}$ обладают свойством симметричности, т.е. $A_{\alpha\beta}=A_{\beta\alpha}$, то из инвариантности 
			\[
				f = A_{\alpha\beta} u^\alpha u^\beta
			\]
			для любого вектора $u^\alpha$ следует, что $A_{\alpha\beta}$ являются компонентами ковариантного тензора.
		}
	\end{exampleblock}
	
}
 
 
\frame{
	\frametitle{ Фундаментальный тензор }
	
	\begin{exampleblock}{Определение}
		\parbox{\textwidth}{
		Выражение
			\begin{equation}
			\label{eq:}
			ds^2 = g_{ik}\argxn dx^i dx^k,
			\end{equation}
		определяющее расстояние между двумя бесконечно близкими точками многообразия, называется \alert{фундаментальной квадратичной формой}.
		}
	\end{exampleblock}
	
}

\frame{
	\frametitle{ Постулаты для фундаментального тензора }
	
	\begin{exampleblock}{Инвариантность}
		\parbox{\textwidth}{
						\centering
			$ds^2$ -- инвариант по определению.
		}
	\end{exampleblock}\pause

	\begin{exampleblock}{Симметричность}
		\parbox{\textwidth}{
			\[
			g_{ij} = g_{ji}
			\]
			
		}
	\end{exampleblock}\pause

	\begin{exampleblock}{Определитель}
		\parbox{\textwidth}{
		\[
		g = \left| 
		\begin{array}{cccc}
		g_{11} & g_{12} & \ldots & g_{1n} \\
		g_{21} & g_{22} & \ldots & g_{2n} \\
		\vdots & \vdots & \ddots & \vdots \\
		g_{n1} & g_{n2} & \ldots & g_{nn} \\
		
		\end{array}
		\right| \neq 0
		\]
			
		}
	\end{exampleblock}\pause

	\begin{exampleblock}{Тензорная природа (следствие)}
		\parbox{\textwidth}{
			\centering
			$g_{ij}$ -- ковариантный тензор по тензорному признаку для симметричного объекта.
		}
	\end{exampleblock}
	
}

\frame{
	\frametitle{ Контравариантный фундаментальный тензор }
	
	\begin{exampleblock}{Определение}
		\parbox{\textwidth}{
			Пусть $A^k$ -- произвольный вектор, тогда рассмотрим вектор
			\[
				A_i = g_{ik} A^k.
			\]
			\pause
			Эти равенства можно рассматривать как систему из $n$ линейных уравнений относительно $A^k$, и т. к. $g \neq 0$, то существуют такие числа $g^{ik}$, что
			\[
				A^i = g^{ik} A_k. 
			\] \pause 
			Так как $A^i$, $A_k$ -- произвольные векторы, то по признаку тензорного объек\-та $g^{ik}$ является контравариантным тензором и называется \alert{конт\-равариантным фундаментальным тензором}.
			
		}
	\end{exampleblock}
	
}

\frame{
	\frametitle{ Смешанный фундаментальный тензор }
	
	\begin{exampleblock}{Определение}
		\parbox{\textwidth}{
			Смешанный тензор 
			\[
				g_i^k = g_{i\alpha}g^{\alpha k}
			\]
			называется \alert{смешанным фундаментальным тензором}.
		}
	\end{exampleblock}\pause

	\begin{exampleblock}{Свойства}
		\parbox{\textwidth}{
			\begin{enum}
				\item равенство $A_i = g_{i\alpha}A^\alpha =  g_{i\alpha} g^{\alpha k}A_k = g_i^k A_k$ имеет место для всех $A_k$, поэтому 
				\[
					g_i^k = \delta_i^k = \left\{
					\begin{array}{cc}
						1, & i = k, \\
						0, & i \neq k.
					\end{array}
					\right.
				\]
				\item
				сокращение смешанного фундаментального тензора по его индексам дает размерность рассматриваемого многообразия, т.к. 
				\[
					g_i^i = \delta_i^i = n.
				\]
			\end{enum}
			
		}
	\end{exampleblock}


	
}
\frame{
	\frametitle{ Переход между контравариантными и ковариантными величинами}
	
	\begin{exampleblock}{Формулы перехода для векторов}
		\parbox{\textwidth}{
			\[
				A_i = g_{ik} A^k,\quad
				A^i = g^{ik} A_k, где			
			\]
			где $A_i$, $A^k$ -- ковариантные и контравариантные компоненты одного и того же вектора.
		}
	\end{exampleblock}\pause

	\begin{exampleblock}{Формулы перехода для тензоров}
		\parbox{\textwidth}{
			\[
			A_{i\cdot}^{\cdot\beta} = A^{\alpha\beta}g_{i\alpha},\quad
			A_{\cdot k}^{\alpha\cdot} = A^{\alpha\beta}g_{\beta k},
			\]\pause
			\[
			A_{ik} = A_{i\cdot}^{\cdot\beta}g_{\beta k} = A^{\alpha\beta}g_{i\alpha}g_{\beta k},
			\]\pause
			\[
			A^{\alpha\beta} = A_{i\cdot}^{\cdot\beta}g^{i\alpha}=A_{ik}g^{k\beta}g^{i\alpha},
			\]
			где $A_{ik}$, $A^{\alpha\beta}$, $A_{i\cdot}^{\cdot\beta}$, $A_{\cdot k}^{\alpha\cdot}$ -- компоненты одного и того же тензора. 
		}
	\end{exampleblock}
	
}

\frame{
	\frametitle{ Связь фундаментальной формы и преобразования координат }
	
	\parbox{\textwidth}{
		
	Пусть подпространство $E_n$ вложено в евклидово пространство $E_m$ ($m \geq n$) и преобразование координат задается 
	\[\left\{
	\begin{array}{ccc}	
		y_1 & = & y_1\argxn,\\
		\vdots & \vdots & \vdots \\
		y_m & = & y_m\argxn,
	\end{array}		
	\right.
	\]
	где $y_\alpha$ -- координаты точек в прямолинейной ортогональной системе координат.
	}\pause

	\medskip
	Тогда фундаментальная квадратичная форма будет иметь вид\\
	\medskip
	\centering
	$ds^2 = \sum\limits_{\alpha=1}^m dy_\alpha^2 = g_{ik} dx^i dx^k$, где
	$g_{ik} = \displaystyle\sum\limits_{\alpha=1}^m \pd{y_\alpha}{x^i} \pd{y_\alpha}{x^k}$.
	
}

\frame{
	\frametitle{ Длина вектора }
	
	\parbox{\textwidth}{
	Рассмотрим контравариантный вектор $A^i$. Подберем бесконечно малый вектор $dx^i$ такой, что
	\[
		dx^i = \lambda A^i, \quad
		A^i = \frac{1}{\lambda} dx^i.
	\]
	
	Так как при преобразовании системы координат компоненты векторов $dx^i$ и $A^k$ преобразуются по одним и тем же формулам, то величина $\lambda$ является инвариантом. \pause
	
	\medskip
	Вектору в подпространстве $R^n$ с координатами $dx^i$ отвечает вектор в пространстве $R^m$ с координатами $y_\alpha$, длина которого равна $ds$. Таким образом,
	\[
		ds^2 = \lambda^2 g_{ik} A^i A^k.
	\]

	}


	
	
}

\frame{
	\frametitle{ Длина вектора }
		\medskip
	\begin{exampleblock}{Определение}
		\parbox{\textwidth}{
			\alert{Длиной вектора} будем называть величину
			\[
			l(A^i) = \sqrt{g_{ik}A^iA^k}.
			\]
		}
	\end{exampleblock}	

	\begin{exampleblock}{Свойства}
		\parbox{\textwidth}{
			\[
				l(A^i) = \sqrt{g_{ik}A^iA^k} = \sqrt{A_iA^i} = \sqrt{g^{ij} A_i A_j}
			\]
			
		}
	\end{exampleblock}

}

\frame{
	\frametitle{ Скалярное произведение векторов }
	
	\parbox{\textwidth}{
	Так как скалярному вектору $dx^i$ отвечает в пространстве вектор с составляющими 
	\[
		dy_\alpha =\pd{y_\alpha}{x^i} dx^i \quad (\alpha=1,\ldots, m),
	\]
	то вектор $A^i=dx^i/\lambda$ в пространстве $R^m$ будет иметь компоненты 
	\[
		a_\alpha = \pd{y_\alpha}{x^i} A^i \quad (\alpha=1,\ldots, m).
	\]
	
	Рассмотрим еще один вектор с компонентами $B^i$, имеющий в $R^m$ компоненты
	\[
	b_\alpha = \pd{y_\alpha}{x^k}B^k \quad (\alpha=1,\ldots, m).
	\]
	}
	
}

\frame{
	\frametitle{ Скалярное произведение векторов }
	
	\begin{exampleblock}{Определение}
		\parbox{\textwidth}{
		\[
			\vec{a}\cdot\vec{b} = \sum\limits_{\alpha=1}^m a_\alpha b_\alpha = 
			\sum\limits_{\alpha=1}^m \pd{y_\alpha}{x^i} A^i \pd{y_\alpha}{x^k} B^k=
			\sum\limits_{\alpha=1}^m \pd{y_\alpha}{x^i} \pd{y_\alpha}{x^k}  A^i B^k=
			g_{ik}A^i B^k
		\]	
		}
	\end{exampleblock}\pause

	\begin{exampleblock}{Свойства:}
		\parbox{\textwidth}{
			
		\begin{enum}
			
		\item
		$
			\vec{a}\cdot\vec{b} = g_{ik}A^i B^k = A_k B^k = A^i B_i = g^{ij} A_i B_j;
		$	\pause
		
		\item
		$
			cos\theta = \displaystyle\frac{	\vec{a}\cdot\vec{b} } {ab}= \frac{g_{ik}A^i B^k}{\sqrt{g_{ik} A^i A^k}\sqrt{g_{ik} B^i B^k}};
		$ \pause
		
		\item
		$
		\vec{a} \perp \vec{b} \Leftrightarrow \vec{a}\cdot\vec{b} = 0 \Leftrightarrow g_{ik}A^i B^k = 0.
		$
		\end{enum}
		}

	\end{exampleblock}
	
}

\frame{
	\frametitle{ Векторное произведение }

	\begin{exampleblock}{Определение}
		\parbox{\textwidth}{
			Пусть  $\boldsymbol{u} = \boldsymbol{A}\times\boldsymbol{B}$, тогда
			\[
			u_i = e^{\alpha\beta\phantom{i}}_{\phantom{\alpha\beta} i} A_\alpha B_\beta = 
			g_{i\gamma}e^{\alpha\beta\gamma}A_\alpha B_\beta=
			\]
			\[
			=	
			\frac{1}{\sqrt{g}}\left\{
			g_{i1}(A_2 B_3-A_3 B_2) + g_{i2}(A_3 B_1 -A_1 B_3) + g_{i3}(A_1 B_2 -A_2B_1)
			\right\}.
			\]
		}
	\end{exampleblock}
	
\begin{exampleblock}{Обозначения}
	\parbox{\textwidth}{
	\[
	e_{\alpha\beta\gamma} = \sqrt{g}\delta_{\alpha\beta\gamma},\quad
	e^{\alpha\beta\gamma} = \frac{1}{\sqrt{g}}\delta_{\alpha\beta\gamma},\quad
	g = |g_{ik}|.
	\]	
	\centering
%	$\delta_{ijk}$ -- тензор, имеющий следующее определение
	\[
	\begin{array}{ccccccc}
	\delta_{123} & = & \delta_{231} & = & \delta_{312} & = & 1,\\
	\delta_{132} & = & \delta_{213} & = & \delta_{321} & = & -1,\\
	\end{array}
	\]
	$\delta_{ijk} =  0$ во всех остальных случаях.
	}
\end{exampleblock}


}



\frame{
	\frametitle{ Литература }
	\begin{literature}[partopsep=1pt,label=]
		\item {\em Кочин~Н.~Е.} Векторное исчисление и начала тензорного исчисления. Изд. 9-е. М.: Наука, 1965.		
	\end{literature}
	
}
	
\end{document}
