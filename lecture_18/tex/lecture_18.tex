\documentclass{beamer}

\usepackage{beamerthemesplit}
\usetheme{Singapore} %Copenhagen}
%\usecolortheme{whale}

%\usepackage[T2A]{fontenc}
%\usepackage[utf8]{inputenc}
%\usepackage[russian]{babel}

\usepackage[main=russian,english]{babel}   %% загружает пакет многоязыковой вёрстки
\usepackage{fontspec}      %% подготавливает загрузку шрифтов Open Type, True Type и др.
\defaultfontfeatures{Ligatures={TeX},Renderer=Basic}  %% свойства шрифтов по умолчанию
\setmainfont{Times New Roman} %% задаёт основной шрифт документа
%\usefonttheme{professionalfonts}% SOLUTION
\usefonttheme{serif}

\usepackage{hyperref}
\usepackage{textcomp}
\usepackage{amssymb,amsmath}
%\usepackage{animate}
%\usepackage{longtable}
\usepackage{xcolor}

%\usepackage{pgffor}
\usepackage{enumitem}
\usepackage[export]{adjustbox}

\newcounter{N}

%% Форматирование окружения itemize
%\usepackage{ragged2e}
%\let\olditem\item
%\renewcommand\item{\olditem\justifying}

\usepackage{ mathrsfs }
\newcommand{\Rho}{\mathscr{P}}

\renewcommand{\Re}{\operatorname{Re}}
\newcommand{\Sh}{\operatorname{Sh}}
\newcommand{\Eu}{\operatorname{Eu}}
\newcommand{\Fr}{\operatorname{Fr}}

%\DeclareMathOperator{\tg}{tg}
\DeclareMathOperator{\сtg}{сtg}


\newcommand{\argxi}{(\xi^1,\xi^2,\xi^3)}
\newcommand{\argx}{(x^1,x^2,x^3)}

\newcommand{\argxiv}{(\vec{\xi})}
\newcommand{\argxv}{(\vec{x})}


\newcommand{\argxbarn}{(\bar{x}^1,\bar{x}^2,\ldots, \bar{x}^n)}
\newcommand{\argxn}{(x^1, x^2,\ldots, x^n)}

\newcommand{\argtxi}{(t, \xi^1,\xi^2,\xi^3)}
\newcommand{\argtoxi}{(t_0, \xi^1,\xi^2,\xi^3)}

\newcommand{\argtxiv}{(t, \vec{\xi})}
\newcommand{\argtoxiv}{(t_0, \vec{\xi})}


\newcommand{\argtx}{(t, x^1,x^2,x^3)}
\newcommand{\argtox}{(t_0, x^1,x^2,x^3)}

\newcommand{\argtxv}{(t, \vec{x})}
\newcommand{\argtoxv}{(t_0, \vec{x})}


\newcommand{\pd}[2]{\frac{\partial #1}{\partial #2}}
\newcommand{\pdk}[2]{\frac{\partial^2 #1}{\partial #2^2}}

\newcommand{\od}[2]{\frac{d #1}{d #2}}
\newcommand{\odk}[3]{\frac{d^{#3} #1}{d #2^{#3}}}

\newcommand{\grad}{\operatorname{grad}}
\newcommand{\rot}{\operatorname{rot}}
\newcommand{\divo}{\operatorname{div}}

\title[]{}

\author[]{ {\em Верещагин Антон Сергеевич}
\\
канд. физ.-мат. наук, старший преподаватель\\
\bigskip
Кафедра аэрофизики и газовой динамики ФФ НГУ}

\usebackgroundtemplate{\includegraphics[width=\paperwidth]{../img/background.png}}

\begin{document}
	
\frame{\titlepage}


\frame{
	\frametitle{Аннотация}
	\parbox{\textwidth}{

	}
}

\frame{
	\frametitle{Элементы термодинамики}
	
	\begin{exampleblock}{Определение}
		\parbox{\textwidth}{
			\alert{Внешними параметрами} называются параметры, определяющие состояние газа только внешними относительно газа телами. (Пример, объем газа, напряжённости полей).
		}
	\end{exampleblock} \pause

	\begin{exampleblock}{Определение}
	\parbox{\textwidth}{
		\alert{Внутренними параметрами} называются параметры, определяющие состояние самого газа. (Например, энергия, давление, температура).
	}
	\end{exampleblock} \pause

	\begin{exampleblock}{Определение}
		\parbox{\textwidth}{
			Состояние газа называется \alert{равновесным}, если оно не изменяется во времени, а также отсутствует обмен энергии с внешними телами. Равновесное состояние -- состояние, из которого газ не может выйти самопроизвольно. Если газ, находящийся в произвольном состоянии, предоставить самому себе, то через некоторое время (\alert{время релаксации}) он перейдёт в равновесное состояние. 
			
		}
	\end{exampleblock}
	
}

\frame{
	\frametitle{Элементы термодинамики }
	
	\begin{exampleblock}{Работа газа}
		\parbox{\textwidth}{
			Работа, совершаемая газом, происходит за счёт изменения внешних параметров $a_i$
			\[
			\delta W = \sum\limits_i A_i da_i,
			\]
			где $A_i$ -- обобщённые силы. 
		}
	\end{exampleblock}\pause

	\begin{exampleblock}{Закон сохранения энергии}
		\parbox{\textwidth}{
			Изменение  внутренней энергии газа $E$ (кинетическая энергия движения молекул и потенциальная энергия их взаимодействия) имеет вид
			\[
			dE = \delta Q - \delta W = \delta Q - \sum\limits_i A_i da_i,
			\]
			где $\delta Q$ -- количество сообщённого газу тепла.
		}
	\end{exampleblock}
	
}

\frame{
	\frametitle{Элементы термодинамики}
	
	\begin{exampleblock}{Уравнения состояния}
		\parbox{\textwidth}{
			По основной теореме термодинамики в равновесном состоянии газа  все внутренние параметры (в качестве которых используются обобщённые силы) являются однозначными функциями внешних параметров  и энергии (или температуры $T$) газа
			\[
			A_i = A_i(T,a_1, \ldots, a_n),\quad
			E = E(T, a_1, \ldots, a_n).
			\]\pause
			Представленные соотношения являются \alert{термическими}  и \alert{калорическим уравнениями состояния}.
			
		}
	\end{exampleblock}
	
}

\frame{
	\frametitle{Элементы термодинамики}
	
	\begin{exampleblock}{Основные предположения}
		\parbox{\textwidth}{
			\begin{enumerate}[label=\arabic*)]
				\item Газ химически и физически однороден.
				\item Отсутствуют внешние поля (гравитационное, магнитное и др.).
				\item Единственные внешним параметром газа является объем $V$, а обобщенной силой -- давление $p$.
			\end{enumerate}	
		Из предположений следует, что многообразие термодинамических состояний \alert{двумерно}.
		}
	

	\end{exampleblock}\pause

	\begin{exampleblock}{Закон сохранения энергии}
		\parbox{\textwidth}{
		\[
			d\varepsilon = \delta Q - p dV,
		\]
		где $V = 1/\rho$ -- удельный объем, $\rho$ -- плотность газа, $\varepsilon$ -- удельная внутренняя энергия газа.
		}
	\end{exampleblock}

	
}


\frame{
	\frametitle{Элементы термодинамики}
	
	\begin{exampleblock}{Второе начало термодинамики}
		\parbox{\textwidth}{
			\[
			dS = \frac{\delta Q}{T} = \frac{1}{T}(d\varepsilon + p dV),
			\]
			где $dS$ -- полный дифференциал от \alert{энтропии} $S=S(V,T)$.

			\pause\medskip
			Таким образом,
			\[
			T dS = d\varepsilon + p dV,
			\]
			для уравнений состояния
			\[
			p=p(V,T),\quad
			\varepsilon = \varepsilon(V,T),\quad
			S = S(V,T).			
			\]
			\pause
			Второе начало термодинамики налагает на уравнения состояния дополнительное условие, поэтому независимых из них всего \alert{два}.
		} 
	\end{exampleblock}
	
}

\frame{
	\frametitle{Элементы термодинамики}
	
	\begin{exampleblock}{}
		\parbox{\textwidth}{
			Из второго начала термодинамики следует, что
			\[
			\pd{S}{T} = \frac{1}{T}\pd{\varepsilon}{T} = \frac{c_V}{T},\quad
			\pd{S}{V} = \frac{1}{T}\left(\pd{\varepsilon}{V} + p\right).
			\]
			
			При заданных уравнениях состояния $\varepsilon = \varepsilon(V,T)$ и $p=p(V,T)$ энтропия находится с точностью до константы, которая исключается с помощью соотношений Нёрста
			\[
			S \to 0 \quad \text{при} \quad T\to 0.
			\]
		}
	\end{exampleblock}
}


\frame{
	\frametitle{Элементы термодинамики}
	
	
	\begin{exampleblock}{Гипотеза о локальном термодинамическом равновесии}
		\parbox{\textwidth}{
			В дальнейшем при изучении течений газа будем считать, что  в каждый момент времени в бесконечно малой окрестности каждой точки пространства газ находится в термодинамически равновесном состоянии и можно ввести понятия
			\[
			p = p\argtxv,\quad
			T = T\argtxv,\quad
			S = S\argtxv,
			\]
			удовлетворяющие термическому, калорическому уравнениям состояния и второму закону термодинамики. 
		}
	\end{exampleblock}
}

\frame{
	\frametitle{Элементы термодинамики}
	
	\begin{exampleblock}{Равновесный процесс}
		\parbox{\textwidth}{
			\[
			\od{S}{t} = \frac{1}{T}\left(\od{\varepsilon}{t} + p \od{V}{t}\right) = \frac{1}{T}	\od{Q}{t},
			\]
			где $\displaystyle\od{Q}{t}$ -- скорость притока тепла к рассматриваемой порции газа.\pause
			
			\medskip
			Если порция газа теплоизолирована $dQ = 0$, тогда равновесный процесс называется \alert{адиабатическим}, для которого
			\[
			\od{S}{t} = 0.
			\]
			
		}
	\end{exampleblock}
	
}

\frame{
	\frametitle{Элементы термодинамики}
	
	\begin{exampleblock}{Неравновесный процесс}
		\parbox{\textwidth}{
			Для теплоизолированной системы
			\[
			\od{S}{t} \geq 0.
			\]\pause
			
			Пусть масса тела участвует в неравновесном процессе, обмениваясь теплом с внешними телами, в этом случе второе начало термодинамики требует выполнения условия
			\[
			\od{S}{t}+\od{S_e}{t} > 0,
			\]
			где $S_e$ -- энтропия внешних тел. Величина $\od{S_e}{t}$ может рассматриваться как поток энтропии от внешних тел к массе тела. 
		}
	\end{exampleblock}
	
}


\frame{
	\frametitle{ Литература }
%	\begin{itemize}[partopsep=1pt,label=\textbullet]
%		\item {\em Кочин~Н.~Е., Кибель~И.~А., Розе~Н.~В.} Теоретическая гидромеханика. М.:Гос. издат. физ.-мат. лит., 1963.
%		
%		\item 
%		{\em Лойцянский~Л.~Г.} Механика жидкости газа и плазмы: Учеб. для вузов. --- 7-е изд., испр. -- М.:Дрофа, 2003
%		
%		\item {\em Бэтчелор Дж.} Введение в динамику жидкости. --- М.: Мир, 1973.
%		
%		\item {\em Ландау~Л.~Д., Лифшиц~Е.~М.} Теоретическая физика: Учебное пособие. В 10 т. Т. VI. Гидродинамика. -- 3-е изд., перераб. --- М.: Наука. Гл. ред. физ-мат. лит., 1986.
%		
%		
%	\end{itemize}
}



\end{document}