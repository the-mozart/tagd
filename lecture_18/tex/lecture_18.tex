\documentclass{beamer}

\usepackage{beamerthemesplit}
\usetheme{Singapore} 

\input{../../include/preamble.inc} 
\input{../../include/definitions.inc} 
\input{../../include/author.inc} 

\title[]{Элементы термодинамики}

\begin{document}
	
\frame[plain]{\titlepage}

\frame[plain]{
	\frametitle{Аннотация}
	\parbox{\textwidth}{
		Элементы термодинамики. Внутренние и внешние параметры. Закон сохранения энергии. Второе начало термодинамики. Совершенный, нормальный и газ Ван-дер-Ваальса. Изоэнтропический и изотермический процессы.
	}
}

\frame{
	\frametitle{Элементы термодинамики}
	
	\begin{exampleblock}{Определение}
		\parbox{\textwidth}{
			\alert{Внешними параметрами} называются параметры, определяющие состояние газа только внешними относительно газа телами (нап\-ример, объем газа, напряженности полей).
		}
	\end{exampleblock} \pause

	\begin{exampleblock}{Определение}
	\parbox{\textwidth}{
		\alert{Внутренними параметрами} называются параметры, определяющие состояние самого газа (например, энергия, давление, температура).
	}
	\end{exampleblock} \pause

	\begin{exampleblock}{Определение}
		\parbox{\textwidth}{
			Состояние газа называется \alert{равновесным}, если оно не изменяется во времени, а также отсутствует обмен энергии с внешними телами. Равновесное состояние -- состояние, из которого газ не может выйти самопроизвольно. Если газ, находящийся в произвольном состоянии, предоставить самому себе, то через некоторое время (\alert{время релаксации}) он перейдет в равновесное состояние. 
			
		}
	\end{exampleblock}
	
}

\frame{
	\frametitle{Элементы термодинамики }
	
	\begin{exampleblock}{Работа газа}
		\parbox{\textwidth}{
			Работа, совершаемая газом, происходит за счет изменения внешних параметров $a_i$:
			\[
			\delta W = \sum\limits_i A_i da_i,
			\]
			где $A_i$ -- обобщенные силы. 
		}
	\end{exampleblock}\pause

	\begin{exampleblock}{Закон сохранения энергии}
		\parbox{\textwidth}{
			Изменение  внутренней энергии газа $E$ (кинетическая энергия движения молекул и потенциальная энергия их взаимодействия) имеет вид
			\[
			dE = \delta Q - \delta W = \delta Q - \sum\limits_i A_i da_i,
			\]
			где $\delta Q$ -- количество сообщенного газу тепла.
		}
	\end{exampleblock}
	
}

\frame{
	\frametitle{Элементы термодинамики}
	
	\begin{exampleblock}{Уравнения состояния}
		\parbox{\textwidth}{
			По основной теореме термодинамики в равновесном состоянии газа  все внутренние параметры (в качестве которых используются обобщенные силы) являются однозначными функциями внешних параметров  и энергии (или температуры $T$) газа:
			\[
			A_i = A_i(T,a_1, \ldots, a_n),\quad
			E = E(T, a_1, \ldots, a_n).
			\]\pause
			Представленные соотношения являются \alert{термическим}  и \alert{калорическим уравнениями состояния}.
			
		}
	\end{exampleblock}
	
}

\frame{
	\frametitle{Элементы термодинамики}
	
	\begin{exampleblock}{Основные предположения}
		\parbox{\textwidth}{
			\begin{enumerate}[label=\arabic*)]
				\item газ химически и физически однороден;
				\item отсутствуют внешние поля (гравитационное, магнитное и др.);
				\item единственным внешним параметром газа является объем $V$, а обобщенной силой -- давление $p$;
			\end{enumerate}	
		Из предположений следует, что многообразие термодинамических состояний \alert{двумерно}.
		}
	

	\end{exampleblock}\pause

	\begin{exampleblock}{Закон сохранения энергии}
		\parbox{\textwidth}{
		\[
			d\varepsilon = \delta Q - p dV,
		\]
		где $V = 1/\rho$ -- удельный объем; $\rho$ -- плотность газа; $\varepsilon$ -- удельная внутренняя энергия газа.
		}
	\end{exampleblock}

	
}


\frame{
	\frametitle{Элементы термодинамики}
	
	\begin{exampleblock}{Второе начало термодинамики}
		\parbox{\textwidth}{
			\[
			dS = \frac{\delta Q}{T} = \frac{1}{T}(d\varepsilon + p dV),
			\]
			где $dS$ -- полный дифференциал от \alert{энтропии} $S=S(V,T)$.

			\pause\medskip
			Таким образом,
			\[
			T dS = d\varepsilon + p dV,
			\]
			для уравнений состояния
			\[
			p=p(V,T),\quad
			\varepsilon = \varepsilon(V,T),\quad
			S = S(V,T).			
			\]
			\pause
			Второе начало термодинамики налагает на уравнения состояния дополнительное условие, поэтому независимых из них всего \alert{два}.
		} 
	\end{exampleblock}
	
}
\frame{
	\frametitle{Элементы термодинамики}
	
	\begin{exampleblock}{Формулы для внутренней энергии}
		\parbox{\textwidth}{
			Из второго начала термодинамики следует, что
			\[
			\pd{}{V}\left(
			\frac{1}{T} \pd{\varepsilon}{T}  
			\right) = \pd{}{T}\left( 
			\frac{1}{T} \pd{\varepsilon}{V}+\frac{p}{T}
			\right) \quad\text{или}\quad
			\pd{\varepsilon}{V} = T^2 \pd{}{T}\left(\frac{p}{T} \right).
			\]
			
			При заданных уравнениях состояния $p = p(V,T)$ и $S=S(V,T)$ внутренняя энергия находится с точностью до константы.
		}
	\end{exampleblock}
}



\frame{
	\frametitle{Элементы термодинамики}
	
	\begin{exampleblock}{Формулы для энтропии}
		\parbox{\textwidth}{
			Из второго начала термодинамики следует, что
			\[
			\pd{S}{T} = \frac{1}{T}\pd{\varepsilon}{T} = \frac{c_V}{T},\quad
			\pd{S}{V} = \frac{1}{T}\left(\pd{\varepsilon}{V} + p\right).
			\]
			
			При заданных уравнениях состояния $\varepsilon = \varepsilon(V,T)$ и $p=p(V,T)$ энтропия находится с точностью до константы, которая исключается с помощью соотношений Нёрста:
			\[
			S \to 0 \quad \text{при} \quad T\to 0.
			\]
		}
	\end{exampleblock}
}


\frame{
	\frametitle{ Внутренняя энергия и энтропия как термодинамические потенциалы }
	
	\begin{exampleblock}{Внутренняя энергия}
		\parbox{\textwidth}{
		Если $\varepsilon = \varepsilon(V, S)$, тогда
		\[
		d\varepsilon = \pd{\varepsilon}{S} dS + \pd{\varepsilon}{V}dV = T dS - p d V \quad \Rightarrow \quad
		T = \pd{\varepsilon}{S},\quad
		p =  - \pd{\varepsilon}{V}.
		\]
		}
	\end{exampleblock}\pause

	\begin{exampleblock}{Энтропия}
		\parbox{\textwidth}{
			Если $S=S(\varepsilon, V)$, тогда
			\[
			\frac{1}{T} = \pd{S}{\varepsilon},\quad
			\frac{p}{T} = \pd{S}{V}.
			\]
		}
	\end{exampleblock}

}

\frame{
	\frametitle{Свободная энергия как термодинамический потенциал}
	
	\begin{exampleblock}{Определение}
		\parbox{\textwidth}{
			Пусть определяющими переменными являются \alert{$V$} и \alert{$T$}, тогда
			\[
			d(\varepsilon - T S) = -S dT - p dV.
			\]
			Если $F = \varepsilon - TS$, тогда
			\[
			S = -\pd{F}{T},\quad
			p = -\pd{F}{V}.
			\]
			
			$F = \varepsilon - TS$ называется \alert{свободной энергией}.
		}
	\end{exampleblock}
	
}

\frame{
	\frametitle{Теплосодержание, или энтальпия}
	
	\begin{exampleblock}{Определение}
		\parbox{\textwidth}{
			Пусть определяющими переменными являются \alert{$p$} и \alert{$S$}, тогда
			\[
			d(\varepsilon + p V) = T dS + V dp.
			\]
			Если $H = \varepsilon + p V$, тогда
			\[
			T = \pd{H}{S},\quad
			V = \pd{H}{p}.
			\]
			
			$H = \varepsilon + pV $ называется \alert{энтальпией}.
		}
	\end{exampleblock}
}


\frame{
	\frametitle{Термодинамический потенциал Гиббса}
	
		\begin{exampleblock}{Определение}
		\parbox{\textwidth}{
			Пусть определяющими переменными являются \alert{$p$} и \alert{$T$}, тогда
			\[
			d(\varepsilon -T S + p V) = -S dT + V dp.
			\]
			Если $G = \varepsilon -T S + p V$, тогда
			\[
			S = -\pd{G}{T},\quad
			V = \pd{G}{p}.
			\]
			
			$G = \varepsilon -T S + pV $ называется \alert{потенциалом Гиббса}.
		}
	\end{exampleblock}
}


\frame{
	\frametitle{Особенности термодинамических потенциалов}
	
	\begin{exampleblock}{Вариативность}
		\parbox{\textwidth}{
		Внутренняя энергия $\varepsilon$ и энтропия $S$ определяются с точностью до аддитивной постоянной, а свободная энергия $F$ и потенциал Гиббса $G$ определяются с точностью до линейной функции от температуры.
		}
	\end{exampleblock}\pause

	\begin{exampleblock}{Свойства среды}
		\parbox{\textwidth}{
			Термодинамические и механические свойства идеальной двухпараметрической среды полностью определяются заданием одной из функций: $\varepsilon(V, S)$, $H(p, S)$, $F(V, T)$, $G(p, T)$.	Для пар переменных $p$ и $V$, $T$ и $S$ нет соответствующих потенциалов.
		}
	\end{exampleblock}

	
}

\frame{
	\frametitle{Удельные теплоемкости $c_p$ и $c_V$}
	
	\begin{exampleblock}{Удельная теплоемкость при постоянном давлении}
		\parbox{\textwidth}{
			\[
			c_p = \left( \vard{Q}{T} \right)_p = 
			\left( \pd{H}{T} \right)_p = 
			\left( \pd{\varepsilon}{T} \right)_p + p\left(\pd{V}{T} \right)_p = 
			\]
			\[
			=
			\left(\pd{\varepsilon}{V}\right)_T \left(\pd{V}{T}\right)_p +
			\left(\pd{\varepsilon}{T}\right)_V + p \left(\pd{V}{T} \right)_p
			\]
		}
	\end{exampleblock}\pause

	\begin{exampleblock}{Удельная теплоемкость при постоянном объеме}
		\parbox{\textwidth}{
			\[
			c_V = \left(\vard{Q}{T}\right)_V = 
			\left(\pd{\varepsilon}{T}\right)_V = 
			\left( \pd{H}{T} \right)_p +
			\left( \pd{H}{p} \right)_T \left(\pd{p}{T}\right)_V -
			V \left(\pd{p}{T} \right)_V
			\]
			
		}
	\end{exampleblock}\pause

	\begin{exampleblock}{Разность теплоемкостей}
	\parbox{\textwidth}{
			\[
			c_p-c_V = 
			\left[
				\left(\pd{\varepsilon}{V}\right)_T  + p 
			\right] 
			\left(\pd{V}{T} \right)_p =
			-\left[
				\left( \pd{H}{p} \right)_T  - V
			\right]
			\left(\pd{p}{T} \right)_V
			\]
	}
	\end{exampleblock}
	

}
%
%\frame{
%	\frametitle{Разность теплоемкостей}
%	
%	
%	
%}


%\frame{
%	\frametitle{$c_p$, $c_v$ и связь между ними}
%	
%	\textbf{Седов~Л.И.} Том. 1. Гл.~V, \S~6.
%	
%}
%
%\frame{
%	\frametitle{ Связь между термодинамическими потенциалами и $c_p$, $c_v$ }
%	
%	\textbf{Седов~Л.И.} Том. 1. Гл.~V, \S~6.
%}


\frame{
	\frametitle{Элементы термодинамики}
	
	
	\begin{exampleblock}{Гипотеза о локальном термодинамическом равновесии}
		\parbox{\textwidth}{
			В дальнейшем при изучении течений газа будем считать, что  в каждый момент времени в бесконечно малой окрестности каждой точки пространства газ находится в термодинамически равновесном состоянии и можно ввести понятия
			\[
			p = p\argtxv,\quad
			T = T\argtxv,\quad
			S = S\argtxv,
			\]
			удовлетворяющие термическому, калорическому уравнениям состояния и второму закону термодинамики. 
		}
	\end{exampleblock}
}

\frame{
	\frametitle{Элементы термодинамики}
	
	\begin{exampleblock}{Равновесный процесс}
		\parbox{\textwidth}{
			\[
			\od{S}{t} = \frac{1}{T}\left(\od{\varepsilon}{t} + p \od{V}{t}\right) = \frac{1}{T}	\od{Q}{t},
			\]
			где $\displaystyle\od{Q}{t}$ -- скорость притока тепла к рассматриваемой порции газа.\pause
			
			\medskip
			Если порция газа теплоизолирована $dQ = 0$, то равновесный процесс называется \alert{адиабатическим}, для него
			\[
			\od{S}{t} = 0.
			\]
			
		}
	\end{exampleblock}
	
}

\frame{
	\frametitle{Элементы термодинамики}
	
	\begin{exampleblock}{Неравновесный процесс}
		\parbox{\textwidth}{
			Для нетеплоизолированной системы:
			\[
			\od{S}{t} \geq 0.
			\]\pause
			
			Пусть масса тела участвует в неравновесном процессе, обмениваясь теплом с внешними телами, в этом случае второе начало термодинамики требует выполнения условия:
			\[
			\od{S}{t}+\od{S_e}{t} > 0,
			\]
			где $S_e$ -- энтропия внешних тел. Величина $\od{S_e}{t}$ может рассматриваться как поток энтропии от внешних тел к массе тела. 
		}
	\end{exampleblock}
	
}

\frame{
	\frametitle{ Совершенный газ }
	
	\begin{exampleblock}{Определение}
		\parbox{\textwidth}{
			\alert{Совершенным (идеальным) газом} называется газ, для которого справедлив закон Менделеева -- Клапейрона:
			\[
			p V = R T,
			\]
			где $R$ -- газовая постоянная.
		}
	\end{exampleblock}





}

\frame{
	\frametitle{ Совершенный газ }
		\begin{exampleblock}{Внутренняя энергия совершенного газа}
		\parbox{\textwidth}{
			Из полученных соотношений для внутренней энергии получаем:		
			\[
			\pd{\varepsilon}{V} = 0 \quad\Rightarrow\quad
			\varepsilon = \varepsilon(T),
			\]
			при этом удельная теплоемкость $c_V= \displaystyle\pd{\varepsilon}{T} = c_V(T)$.
		}
	\end{exampleblock}\pause
	
	\begin{exampleblock}{Определение}
		\parbox{\textwidth}{
			Газ называется \alert{политропным}, если  $c_V$ не зависит от $T$. В этом случае:
			\[
			\varepsilon = c_V T.
			\]
		}
	\end{exampleblock}
}

\frame{
	\frametitle{Совершенный газ}
	
	\begin{exampleblock}{Выводы кинетической теории}
		\parbox{\textwidth}{
			Выражения для удельных теплоемкостей имеют вид
			\[
			c_V = \frac{f}{2}k\frac{N_A}{M},\quad
			R = c_p-c_V,
			\]
			где $f$ -- число степеней свободы молекулы газа ($f=3$ для одноатомного, $f=5$ для двухатомного); $k$ -- постоянная Больцмана; $N_A$ -- постоянная Авогадро; $M$ -- молекулярный вес.
		}
	\end{exampleblock}\pause

	\begin{exampleblock}{Энтропия}
		\parbox{\textwidth}{
			Из соотношения для энтропии при заданных уравнениях состояния следует, что 
			\[
			S = c_V \ln T + R \ln V + const = c_V \ln T + c_p \ln V - c_V \ln V  + const = 
			\]
			\[
			=
			c_V \ln p + c_p \ln V + const.			
			\]
		}
	
	\end{exampleblock}
}

\frame{
	\frametitle{Газ Ван-дер-Ваальса}
	
	\begin{exampleblock}{Уравнение состояния}
		\parbox{\textwidth}{
			Поправка к уравнению состояния идеального газа, связанная с учетом объема молекул и сил молекулярного взаимодействия, приводит к уравнению состояния \alert{Ван-дер-Ваальса}:
			\[
			p = \frac{RT}{V-b} - \frac{a}{V^2},
			\]
			где $a$ -- величина пропорциональная силе сцепления молекул; $b$ -- величина пропорциональная собственному объему молекул газа.
			
		}
	\end{exampleblock}\pause

	\begin{exampleblock}{Внутренняя энергия и энтропия}
		\parbox{\textwidth}{
			\[
			\varepsilon = \int c_V(T) dT - \frac{a}{V},\quad
			S = \int \frac{c_V(T)}{T} dT + R \ln (V-b) + const.
			\]
			
		}
	\end{exampleblock}
	
}

\frame{
	\frametitle{Изоэнтропический процесс}
	
	\begin{exampleblock}{Определение}
		\parbox{\textwidth}{
			
			Если некоторый элемент газа подвергается медленному расширению или сжатию так, что при этом не происходит теплообмена с окружающей средой, то элемент совершает адиабатический переход из одного термодинамического состояния в другое. При этом медленный процесс остается обратимым и энтропия элемента остается  неизменной. Такой переход называется \alert{изоэнтропическим}.
			Кривая $S = const$ называется адиабатой Пуассона. 
			
		}
	\end{exampleblock}\pause
	
	\begin{exampleblock}{Уравнение состояния идеального изоэнтропического политропного газа}
		\parbox{\textwidth}{
					\[
		p = \frac{A}{\gamma}V^{-\gamma},
		\]
		где 
		\[
		\gamma = \frac{c_p}{c_V} = 1 + \frac{R}{c_V} > 1,\quad
		A^2 = A^2(S) = a^2 e^{S/c_V} = const.
		\]
			
			
		}
	\end{exampleblock}
}

\frame{
	\frametitle{Изотермический процесс}
	
	\begin{exampleblock}{Определение}
		\parbox{\textwidth}{
			
			Если некоторый элемент газа подвергается медленному расширению или сжатию так, что при этом не происходит изменения температуры газа, то такой процесс называется \alert{изотермическим}.
			Кривая $T = const$ называется изотермой. 
			
		}
	\end{exampleblock}
	
	\begin{exampleblock}{Уравнение состояния идеального изотермического политропного газа}
		\parbox{\textwidth}{
			\[
			p = c^2\frac{1}{V} = c^2 \rho,
			c^2 = (c_p-c_V)T = RT = const
			\]
			В некоторых случаях идеальный изотермический газа можно рассматривать как политропный газ с показателем $\gamma=1$.
			
		}
	\end{exampleblock}
}

\frame{
	\frametitle{ Нормальный газ }
	
	\begin{exampleblock}{Определение}
		\parbox{\textwidth}{
			Газ называется \alert{нормальным}, если выполнены следующие условия:
			\begin{enum}
			\item
			$
			\displaystyle\pd{p(V,S)}{V} < 0
			$;
			\item
			$
			\displaystyle\pdk{p(V,S)}{V} > 0
			$;	
			\item
			$p(V,S) \to \infty$ при $V\to 0$;
	 		
			\item
			$
			\displaystyle\pd{p(V,S)}{S} > 0
			$;
			\item
			$
			c_V = \displaystyle\pd{\varepsilon(V,T)}{T} > 0
			$;
			\item область переменных $(V,T)$, в которых выполнены условия 1-5, является выпуклой.
			
			\end{enum}
		}
	\end{exampleblock}
}


\frame{
	\frametitle{ Литература }
	\begin{literature} %[partopsep=1pt,label=\textbullet]
	\item
	{\em Базаров И.\,П.} Термодинамика. Учеб. для вузов. -- 4-е изд., перераб. и доп. --- М.: Высш. шк., 1991. 
		
	\item
	{\em Рождественский Б.\,Л., Яненко Н.\,Н.} Системы квазилинейных уравнений и их приложения к газовой динамике. Изд. 2-е, Главная редакция физ.-мат. лит. Изд. <<Наука>>, М., 1978.
		
	\item 
	{\em Седов~Л.\,И.}  Механика сплошной среды. Том\,1. Гл.\,V, \S\,6. 1970.
		

	\end{literature}
}



\end{document}